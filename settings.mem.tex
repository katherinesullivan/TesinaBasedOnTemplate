\documentclass[a4paper, 11pt]{memoir}

\usepackage{helvet}
\renewcommand*{\familydefault}{\rmdefault}

% Enumera las subsections
\setsecnumdepth{subsection}
\setcounter{tocdepth}{2} % pone las subsections en el índice

\usepackage{tocloft} %% Editar el ToC
% \setlength{\cftbeforesectionskip}{0.5em}
% % \setlength{\cftbeforesubsectionskip}{1in}
% \setlength{\cftbeforechapterskip}{1em}
% %\setlength{\cftafterskip}{1em}
% %\setlength{\cftparskip}{0em}
% \setlength{\parindent}{1.5em}
% % \setlength{\parskip}{0.3em}
\cftpagenumberson{subsection}

\usepackage[utf8]{inputenc}
\usepackage[spanish]{babel}
%\usepackage[english]{babel}
\usepackage{paralist}
\usepackage[hidelinks]{hyperref}
\usepackage{array,multirow}
\usepackage{makecell}
\usepackage{adjustbox}
\renewcommand{\cellalign}{cc}
\usepackage[referable]{threeparttablex}
% \usepackage[toc,page]{appendix}
\usepackage{csquotes}

\linespread{1.25}
\setlength{\parskip}{\baselineskip}

% \usepackage{minted}

\usepackage[style=numeric, maxbibnames=10, giveninits=true, uniquename=false, backend=biber, sortcites=true, sorting=none]{biblatex}
\DeclareLanguageMapping{english}{english-apa}
\addbibresource{citas.bib}

\usepackage{todonotes}
\usepackage{stmaryrd}
\usepackage{graphicx,xcolor}
\usepackage{subcaption}
\graphicspath{{Imgs/}{Figures/Plots/multiple/}{Figures/Plots/single/}{Figures/}}
\usepackage{amssymb} %simbolos math
\usepackage{amsmath}
\usepackage{pifont} % simbolos extras
\usepackage{mathtools}

\usepackage{mathrsfs} % \mathsrc
\usepackage{color,soul} % highlighting
\usepackage{commath} % \abs
\usepackage{enumitem} % specify labels in enums
\usepackage{tcolorbox} % boxes
\newtcolorbox[auto counter, number within=section]{boxamarillo}[2][]{%
  colback=white,   % Color de fondo
  colbacktitle=yellow, % Color de fondo del título
  coltitle=black,  % Color del título
  title=#1,        % Título del box
  width=\textwidth-1.5em, % Ancho del box
  before=\vspace{0.25cm}, % Espacio vertical antes del box
  after=\vspace{0.25cm},
  #2               % Contenido del box
}

%\renewcommand{\refname}{Referencias} % Cambiar nombre a Referencias
%\bibliographystyle{ieeetr} % Ordenar referencias por orden de aparición

\definecolor{coolgreen}{RGB}{64, 168, 80}
\definecolor{coolred}{RGB}{192, 57, 43} 
\definecolor{coolblue}{RGB}{52, 152, 219}   % Muted blue
\definecolor{coolorange}{RGB}{230, 126, 34} % Soft orange
\definecolor{coolpurple}{RGB}{155, 89, 182} % Calm purple
\newcommand{\xmark}{{\color{coolred}{\ding{55}}}}
\newcommand{\cmark}{{\color{coolgreen}{\ding{51}}}}

%lemmas
\usepackage{amsthm}
% \newtheorem{theorem}{Teorema}[section]
% \newtheorem{corollary}{Corolario}[theorem]
% \newtheorem{lemma}{Lema}[theorem]

%algorithms
\usepackage{algorithm}
\usepackage{algpseudocode}

\newcommand\mdoubleplus{\ensuremath{\mathbin{+\mkern-10mu+}}}
\newcommand{\zn}[1]{{\protect\color{red}\sf\small #1}}
\newcommand{\za}{\leftarrow}
\newcommand{\zb}{\medskip\noindent$\bullet$\ }
\newcommand{\zdu}{\_\hspace{0.2mm}\_}
\newcommand{\z}[1]{{\scriptsize{\tt #1}:\ \ }}
\newfloat{alg}{th}{tmp}

\newcounter{znoc}
\newcommand{\zno}[1]{\refstepcounter{znoc}\label{#1}\textbf{\ref{#1}}}

\newcommand{\zalg}[1] %TODO poner más espacio entre líneas de algoritmo
{
\begin{figure}[ht]
    % \centering
    % \includegraphics{}
    % \caption{Caption}
    % \label{fig:my_label}

% \vspace{1.5em}
    % \linespread{1.3}
    % \renewcommand{\baselinestretch}{2.0}
\noindent\centerline{
\fbox{
    \normalsize
    \begin{minipage}[l]{8cm}
        \begin{tabbing}
        \textbf{Algorithm} #1
        \end{tabbing}
    \end{minipage}
}}
% \vspace{1.5em}

\end{figure}{}
}

% \setlength{\abovedisplayskip}{1em}
% \setlength{\belowdisplayskip}{1em}

% \newcommand{\zalg}[1]{
% {\begin{alg}
% \footnotesize\noindent\centerline{\fbox{%
% % \footnotesize\bigskip\noindent\centerline{\fbox{%
% \begin{minipage}[l]{8cm}\begin{tabbing}
% \textbf{Algorithm} #1
% \end{tabbing}\end{minipage}}}\end{alg}}}


% Para graficar
\usepackage{tikz}
\usepackage{etoolbox}
\usetikzlibrary{ arrows.meta, positioning, shapes.geometric}
% \tikzset{
%     block/.style =  { draw
%                     , thick
%                     , rectangle
%                     , fill=green!10},
%     dot/.style  =   { draw
%                     , circle
%                     , minimum size = .2em
%                     , inner sep = 0pt
%                     , outer sep = 0pt
%                     , fill=black
%                 },
%     }
\AtBeginEnvironment{tikzpicture}{\shorthandoff{>}\shorthandoff{<}}{}{}

% PP Haskell
\usepackage{listings}
\usepackage{color}
%\lstset{language=Haskell}

% \lstdefinestyle{customhaskell}{
%     breaklines=true,
%     %frame=L,
%     %xleftmargin=\parindent,
%     language=Haskell,
%     showstringspaces=false,
%     %showstringspaces=true,
%     basicstyle=\footnotesize\rmfamily,
%     keywordstyle=\bfseries\color{green!40!black},
%     commentstyle=\itshape\color{blue!40!black},
%     identifierstyle=\color{blue},
%     stringstyle=\color{orange!40!black},
% }
% \lstset{language=Haskell,style=customhaskell}
% \lstset{
%     linewidth=\textwidth,
%     inputencoding=utf8,
%     extendedchars=true,
%     morekeywords={mkVar,mkVars, par, pseq, pars, seqs, TPar,smap},
%     literate=%
%              {→}{{$\rightarrow$}}2
%              {->}{{$\rightarrow$}}2
%              {<-}{{$\leftarrow$}}2
%              {í}{{\'{i}}}1
%              {ó}{{\'{o}}}1
%              {é}{{\'{e}}}1
%              {á}{{\'{a}}}1
%              {ú}{{\'{u}}}1
% }

% \lstnewenvironment{code}[1][]
%     {\minipage{\textwidth}}
%     {\endminipage}

\newcommand{\HRule}{\rule{\linewidth}{0.5mm}}
% \newcommand{\haskell}[1]{\lstinline[breaklines=false]{#1}}
% \newcommand{\phaskell}[1]{\lstinline{#1}}
% \newcommand{\rhaskell}[1]{%
% \lstinline[keywordstyle=\bfseries\color{red},identifierstyle=\color{red},basicstyle=\footnotesize\rmfamily\color{red}]{#1}%
% }

% \usepackage{float}
% \floatstyle{plain}
% \newfloat{program}{t!}{loe}
% \floatname{program}{C\'odigo}

%%%%%%%%%%%%%%%%%%%%%%%% Nombres...
% \newcommand{\nombre}{\textit{Klytius}}
%%%%%%%%%%%%%%%%%%%%%%%% Nombres...

% \makeatletter
% %\AtBeginDocument{%
% \let\c@figure\c@table
% \let\thefigure\thetable
% \let\ftype@table\ftype@figure% give the floats the same precedence
% \let\c@program\c@table
% \let\theprogram\thetable
% \let\ftype@program\ftype@table% give the floats the same precedence
% %\let\c@table\c@program
% %\let\thetable\theprogram
% %\let\ftype@program\ftype@table% give the floats the same precedence
% %\let\c@program\c@figure
% %\let\theprogram\thefigure
% %\let\ftype@figure\ftype@program% give the floats the same precedence
% %}
% \makeatother

\newcommand{\comentario}[1]{\todo[color=green!50]{C.: #1}}
\newcommand{\seba}[1]{\todo[color=red!50]{S.: #1}}

\newenvironment{dedication}
    {\vspace{6ex}\begin{quotation}\begin{center}\begin{em}}
    {\par\end{em}\end{center}\end{quotation}}


% Comandos Kathy

% % Comandos teoremas
\newtheorem{theorem}{Teorema}[section]
\newtheorem{corollary}{Corolario}[theorem]
\newtheorem{lemma}[theorem]{Lema}
\newtheorem{definition}{Definición}[section]
\newtheorem{proposition}{Proposición}
\newtheorem{example}{Ejemplo}[section]

% Comandos
\newcommand{\St}{\mathcal{S}}
\newcommand{\K}{\mathcal{K}}
\newcommand{\Gk}{\mathcal{G}_\K}
\newcommand{\Hk}{\mathcal{H}_\K}
\newcommand{\Act}{\mathcal{A}}
\newcommand{\Prob}{\mathbb{P}}

\newcommand{\G}{\mathcal{G}}
\newcommand{\Gl}{\mathcal{G}^\mathit{l}}

\newcommand{\cuad}{\square}
\newcommand{\diam}{\diamondsuit}
\newcommand{\picuad}{\pi_\cuad}
\newcommand{\Picuad}{\Pi_\cuad}
\newcommand{\pidiam}{\pi_\diam}
\newcommand{\Pidiam}{\Pi_\diam}

\newcommand{\piHdiamMD}{\pidiam \in \Pi^{MD}_{\Hk,\diam}}
\newcommand{\piHcuadMD}{\picuad \in \Pi^{MD}_{\Hk,\cuad}}
\newcommand{\piHdiamS}{\pidiam \in \Pi^{S}_{\Hk,\diam}}
\newcommand{\piHcuadS}{\picuad \in \Pi^{S}_{\Hk,\cuad}}
\newcommand{\piHdiamXS}{\pidiam \in \Pi^{XS}_{\Hk,\diam}}
\newcommand{\piHcuadXS}{\picuad \in \Pi^{XS}_{\Hk,\cuad}}
\newcommand{\piHdiam}{\pidiam \in \Pi_{\Hk,\diam}}
\newcommand{\piHcuad}{\picuad \in \Pi_{\Hk,\cuad}}

\newcommand{\piGdiamMD}{\pidiam \in \Pi^{MD}_{\Gk,\diam}}
\newcommand{\piGcuadMD}{\picuad \in \Pi^{MD}_{\Gk,\cuad}}
\newcommand{\piGdiamS}{\pidiam \in \Pi^{S}_{\Gk,\diam}}
\newcommand{\piGcuadS}{\picuad \in \Pi^{S}_{\Gk,\cuad}}
\newcommand{\piGdiamXS}{\pidiam \in \Pi^{XS}_{\Gk,\diam}}
\newcommand{\piGcuadXS}{\picuad \in \Pi^{XS}_{\Gk,\cuad}}
\newcommand{\piGdiam}{\pidiam \in \Pi_{\Gk,\diam}}
\newcommand{\piGcuad}{\picuad \in \Pi_{\Gk,\cuad}}
\newcommand{\pidiamset}{\pidiam \in \Pidiam}
\newcommand{\picuadset}{\picuad \in \Picuad}

\newcommand{\ProbH}{\Prob_{\Hk, s}^{\picuad, \pidiam}}
\newcommand{\ProbG}{\Prob_{\Gk, s}^{\picuad, \pidiam}}

\newcommand{\siempevent}{\cuad \diam}
\newcommand{\eventsiemp}{\diam \cuad}
\newcommand{\alc}{\diam}
\newcommand{\nextltl}{\bigcirc}

\newcommand{\EC}{\mathcal{E}}
\newcommand{\newV}{\bigcup_{\substack{s \in \St \\ V' \in V_s}} v_{V'}}
\newcommand{\DGk}{Derand(\Gk)}
\newcommand{\W}{\mathcal{W}}

\newcommand{\picuadGk}{\hat{\pi}_{\cuad}^{*}}
\newcommand{\picuadDGk}{\check{\pi}_{\cuad}^{*}}

\newcommand{\infsigma}{\text{Inf}(\sigma)}

%\newcommand{\val}[1]{\sup_{\picuad \in \Picuad} \inf_{\pidiam \in \Pidiam} \Prob_{\Gk, #1}^{\picuad, \pidiam} (\phi)}

\newcommand{\until}{\ \mathcal{U} \ }

\newcommand{\M}{\mathcal{M}}
\newcommand{\ProbPMDP}{\Prob_{\M, s}^\pi}

\newcommand{\inft}{\textit{inf}}

\newcommand{\paths}{\textit{Paths}}

\newcommand{\pathsfin}{\textit{Paths}_\textit{fin}}

\newcommand{\R}{\mathbb{R}}
\newcommand{\N}{\mathbb{N}}

\newcommand{\rand}{\textsf{rand}}
\newcommand{\derand}{\textsf{derand}}
\newcommand{\supp}{\textsf{supp}}
\DeclareRobustCommand{\hlcyan}[1]{{\sethlcolor{cyan}\hl{#1}}}

\newcommand{\eventE}{\mathfrak{E}}
\newcommand{\B}{\mathfrak{B}}
\newcommand{\dist}{\mathsf{Dist}}
\newcommand{\val}{\textit{Val}}
\newcommand{\V}{\mathbb{V}}