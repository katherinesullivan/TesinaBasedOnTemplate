\chapter{Objetivos $\omega$-regulares}
~\label{cap:objetivos}

En este capítulo presentaremos la formalización de los objetivos
$\omega$-regulares, hablaremos un poco sobre cómo surgen, veremos una
clasificación típica de los objetivos $\omega$-regulares viendo ejemplos de
cada uno y desarrollaremos sobre la importancia de los objetivos de Rabin y
nuestro interés en su estudio.

\section{Formalización de objetivos $\omega$-regulares}

Los objetivos $\omega$-regulares nacen de la extensión de la teoría de
lenguajes regulares al dominio de las palabras infinitas. Mientras que los
lenguajes regulares clásicos describen conjuntos de cadenas finitas
reconocibles por autómatas finitos, los lenguajes $\omega$-regulares operan
sobre secuencias infinitas, capturando comportamientos infinitos. Esta
generalización fue formalizada en los años sesenta con los autómatas de Büchi,
que introdujeron condiciones de aceptación sobre corridas infinitas de
autómatas de estado finito.

Los objetivos $\omega$-regulares surgen principalmente en el contexto de la
verificación y síntesis de sistemas reactivos, donde es necesario especificar y
razonar sobre comportamientos que pueden durar indefinidamente. Estos objetivos
permiten expresar propiedades como "algo bueno ocurre infinitamente a menudo" o
"algo malo ocurre sólo una cantidad finita de veces".

Existen varios tipos de objetivos $\omega$-regulares, cada uno definido por
diferentes condiciones de aceptación en los autómatas sobre palabras infinitas.
Los más conocidos son:

\section{Clasificación de objetivos $\omega$-regulares}

\begin{itemize}
	\item \textbf{Objetivo de Büchi:} Un autómata de Büchi acepta una corrida infinita si visita al menos un estado aceptante infinitamente a menudo. \emph{Ejemplo:} "El sistema debe responder a cada petición infinitas veces".
	\item \textbf{Objetivo de co-Büchi:} Se acepta si sólo se visitan finitamente muchos estados no aceptantes. \emph{Ejemplo:} "Después de algún punto, el sistema nunca entra en un estado de error".
	\item \textbf{Objetivo de Paridad:} Cada estado tiene asignado un color (número natural), y se acepta si el mínimo color que aparece infinitamente a menudo es par. \emph{Ejemplo:} "Las prioridades más bajas (pares) prevalecen en el comportamiento infinito".
	\item \textbf{Objetivo de Rabin:} Definido por un conjunto de pares $(E_i, F_i)$; se acepta si existe un $i$ tal que se visitan infinitamente a menudo estados en $F_i$ y sólo finitamente a menudo estados en $E_i$. \emph{Ejemplo:} "Cierta condición deseada ocurre infinitamente a menudo, mientras que una condición prohibida ocurre sólo finitamente".
	\item \textbf{Objetivo de Streett:} Dual al de Rabin; se acepta si para todo par $(E_i, F_i)$, si $E_i$ ocurre infinitamente a menudo, entonces también $F_i$ ocurre infinitamente a menudo.
\end{itemize}

Los objetivos de Rabin son especialmente importantes porque cualquier lenguaje
$\omega$-regular puede ser reconocido por un autómata de Rabin. Además, los
autómatas de Rabin son fundamentales en la teoría de juegos y síntesis de
controladores, ya que permiten expresar una amplia gama de especificaciones y
tienen buenas propiedades de cierre bajo operaciones booleanas.

¿de dónde salen?

¿cuáles son los tipos de objetivos omega regulares más conocidos? agg 1 ejemplo de cada uno

¿por qué los objetivos de Rabin son importantes?

\section{La importancia de los objetivos de Rabin}

