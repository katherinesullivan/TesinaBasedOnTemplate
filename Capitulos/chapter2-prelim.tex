\chapter{Preliminares}
~\label{cap:pre}

\section{sigma algebras}
~\label{cap:pre:sec:algebra}

\subsection{Baier y Katoen}

Una \textbf{\(\sigma\)-álgebra} es un par \((Outc, \eventE)\) donde \(Outc\) es
un conjunto no vacío y \(\eventE \subseteq 2^{Outc}\) es un conjunto que
consiste en subconjuntos de \(Outc\) que contiene el conjunto vacío y es
cerrado bajo complementación y uniones contables, es decir:

\begin{itemize}
	\item \(\emptyset \in \eventE\),
	\item si \(E \in \eventE\), entonces \(Outc \setminus E \in \eventE\),
	\item si \(E_1, E_2, \ldots \in \eventE\), entonces \(\bigcup_{n=1}^{\infty} E_n \in
	      \eventE\).
\end{itemize}

Nótese que las condiciones de las \(\sigma\)-álgebras implican que \(Outc \in
\eventE\), ya que \(Outc \neq \emptyset\), y que \(\eventE\) es cerrado bajo
intersecciones contables, ya que \(\bigcap_{n=1}^{\infty} E_n =
\overline{\bigcap_{n=1}^{\infty} \overline{E_n}}\).

A veces se supone que el conjunto \(Outc\) está fijado y \(\eventE\) se llama
una \(\sigma\)-álgebra. Los elementos de \(Outc\) a menudo se llaman resultados
(\textit{outcomes}), mientras que los elementos de \(\eventE\) se llaman
eventos.

Para cualquier conjunto \(Outc\), el conjunto potencia \(\eventE = 2^{Outc}\)
produce una \(\sigma\)-álgebra sobre \(Outc\). En esta \(\sigma\)-álgebra,
todos los subconjuntos de \(Outc\) son eventos. El otro extremo es la
\(\sigma\)-álgebra que consiste en el conjunto vacío y \(Outc\), es decir,
\(\eventE = \{ \emptyset, Outc \}\). Aquí, ningún subconjunto propio no vacío
de \(Outc\) es un evento.

Una\textbf{ medida de probabilidad} sobre \((Outc, \eventE)\) es una función
\(Pr : \eventE \to [0, 1]\) tal que \(Pr(Outc) = 1\), y si
\((E_n)_{n=1}^{\infty}\) es una familia de eventos disjuntos dos a dos con
\(E_n \in \eventE\), entonces:

\[
	Pr\left(\bigcup_{n=1}^{\infty} E_n\right) = \sum_{n=1}^{\infty} Pr(E_n).
\]

Un \textbf{espacio de probabilidad} es una \(\sigma\)-álgebra equipada con una
medida de probabilidad, es decir, es un trupla \((Outc, \eventE, Pr)\) donde
\((Outc, \eventE)\) es una \(\sigma\)-álgebra y \(Pr\) es una medida de
probabilidad sobre \((Outc, \eventE)\). El valor \(Pr(E)\) se llama la medida
de probabilidad de \(E\), o simplemente la probabilidad de \(E\).

En el contexto de las medidas de probabilidad, los eventos (es decir, los
elementos de \(\eventE\)) a menudo se dice que son medibles. Es decir, la
medibilidad de un conjunto \(E \subseteq Outc\) significa que \(E \in
\eventE\), y por lo tanto, tiene sentido hablar de la medida de probabilidad de
\(E\).

En general, siempre que \(Outc\) sea contable, se puede obtener una medida de
probabilidad sobre el conjunto potencia de \(Outc\) fijando una función \(\mu :
Outc \to [0, 1]\) tal que

\[
	\sum_{e \in Outc} \mu(e) = 1.
\]

Tales funciones \(\mu\) se llaman \textbf{distribuciones} sobre \(Outc\).
Cualquier distribución \(\mu\) induce una medida de probabilidad sobre la
\(\sigma\)-álgebra \(E = 2^{Outc}\) de la siguiente manera. Para un subconjunto
\(E\) de \(Outc\), \(Pr_\mu(E)\) se define como

\[
	Pr_\mu(E) = \sum_{e \in E} \mu(e).
\]

De hecho, es fácil verificar que \(\mu\) satisface las condiciones de una
medida de probabilidad. En lo sucesivo, \(Pr_\mu(E)\) a menudo se abrevia como
\(\mu(E)\) y \(Distr(Outc)\) se usa para denotar el conjunto de distribuciones
sobre \(Outc\).

Resumamos algunas propiedades fundamentales de las medidas de probabilidad.
Dado que \(E \cup \overline{E} = Outc\) y \(E\) y \(\overline{E}\) son
disjuntos, las condiciones anteriores implican que

\[
	Pr(E) = 1 - Pr(\overline{E}).
\]

En particular, \(Pr(\emptyset) = Pr(Outc) = 1 - Pr(Outc) = 1 - 1 = 0\).

Las medidas de probabilidad son monótonas, es decir, para eventos \(E\) y
\(E'\) tales que \(E \subseteq E'\), se cumple que

\[
	Pr(E') = Pr(E') + Pr(E' \setminus E) \geq Pr(E).
\]

Además, si \((E_n)_{n=1}^{\infty}\) es una familia de eventos, posiblemente no
disjuntos entre sí, entonces:

\[
	\bigcup_{n=1}^{\infty} E_n = \bigcup_{n=1}^{\infty} E'_n,
\]

donde \(E'_1 = E_1\) y \(E'_n = E_n \setminus (E_1 \cup \ldots \cup E_{n-1})\)
para \(n \geq 2\). Dado que \(E'_n \cap E'_m = \emptyset\) para \(n \neq m\),
tenemos que:

\[
	Pr\left( \bigcup_{n=1}^{\infty} E_n \right) = Pr\left( \bigcup_{n=1}^{\infty} E'_n \right) = \sum_{n=1}^{\infty} Pr(E'_n).
\]

Si \(E_1 \subseteq E_2 \subseteq E_3 \subseteq \cdots\) y \(E'_n\) es como se
indicó antes, entonces tenemos \(E'_n = E_n \setminus E_{n-1}\) para \(n \geq
2\), lo que resulta en:

\begin{align*}
	&Pr\left( \bigcup_{n=1}^{\infty} E_n \right) = Pr(E_1) + \sum_{n=2}^{\infty} (Pr(E_n) - Pr(E_{n-1})) = \\ &= \lim_{N \to \infty} \left( Pr(E_1) + \sum_{n=2}{N} Pr(E_n) - Pr(E_{n-1})\right) =\lim_{N \to \infty} Pr(E_N).
\end{align*}

Este límite existe y concuerda con el supremo de \(\{Pr(E_1), Pr(E_2), \dots
\}\), ya que la monotonía de \(Pr\) implica que \(Pr(E_1) \leq Pr(E_2) \leq
\dots \leq 1\). Para intersecciones numerables se aplican resultados análogos.
Es decir, si \(E_1 \supseteq E_2 \supseteq \dots\), entonces

\[
	Pr\left( \bigcap_{n=1}^{\infty} E_n \right) = \lim_{n \to \infty} Pr(E_n) = \inf_{n \geq 1} Pr(E_n).
\]

Esto se sigue del hecho de que la secuencia \((\overline{E_n})_{n \geq 1}\) de
los complementos \(\overline{E_n} = \text{Outc} \setminus E_n\) es decreciente.
Por lo tanto, obtenemos:

\[
	Pr\left( \bigcap_{n=1}^{\infty} E_n \right) = 1 - Pr(\overline{\bigcup_{n \geq 1} \overline{E_n}}) = 1 - \lim_{n \to \infty} Pr(\overline{E_n}) = 1 - \lim_{n \to \infty} Pr(\overline{E_n}) = \lim_{n \to \infty} Pr(E_n).
\]

Cualquier evento \(E\) con \(Pr(E) = 1\) se dice que ocurre \textbf{casi
	seguramente}. Nótese que si \(E\) ocurre casi seguramente, entonces \(Pr(D) =
Pr(E \cap D)\) para todos los eventos \(D\), ya que \(Pr(D \setminus E) = 0\),
dado que \(D \setminus E\) es un subconjunto de \(\overline{E}\) y
\(Pr(\overline{E}) = 1 - Pr(E) = 1 - 1 = 0\). Esto implica que:

\[
	Pr(D) = Pr(E \cap D) + Pr(D \setminus E) = Pr(E \cap D).
\]

En particular, el evento \(E_1 \cap E_2\), de eventos \(E_1\) y \(E_2\) que
ocurren casi seguramente, también ocurre casi seguramente. Como se puede
demostrar por inducción, esto se extiende a cualquier evento que pueda
escribirse como una intersección finita \(\bigcap_{1 \leq i \leq n} E_i\) de
eventos \(E_1, \dots, E_n\) que ocurren casi seguramente. Al tomar el límite de
tales intersecciones finitas, obtenemos que \(\bigcap_{i \geq 1} E_i\) ocurre
casi seguramente si \(Pr(E_i) = 1\) para todo \(i \geq 0\).

Para cada conjunto \(\text{Outc}\) y cada subconjunto \(\Pi\) de
\(2^{\text{Outc}}\), existe la \(\sigma\)-álgebra más pequeña que contiene a
\(\Pi\). Esto se debe a las siguientes observaciones:

\begin{itemize}
	\item El conjunto potencia \(2^{\text{Outc}}\) de \(\text{Outc}\) es una
	      \(\sigma\)-álgebra, y
	\item La intersección de \(\sigma\)-álgebras es una \(\sigma\)-álgebra.
\end{itemize}

La intersección \(\eventE_\Pi = \bigcap_\eventE \eventE\), donde \(\eventE\)
varía sobre todas las \(\sigma\)-álgebras sobre \(\text{Outc}\) que contienen a
\(\Pi\), es una \(\sigma\)-álgebra y está contenida en cualquier
\(\sigma\)-álgebra \(\eventE\) tal que \(\Pi \subseteq \eventE\). A esta
\(\sigma\)-álgebra \(\eventE_\Pi\) se la llama la \textbf{\(\sigma\)-álgebra
	generada por \(\Pi\)}, y \(\Pi\) es la \textbf{base} para \(\eventE_\Pi\).

\subsection{Kucera}

\textbf{Espacios de probabilidad}

Sea \( A \) un conjunto finito o numerablemente infinito. Una
\textbf{distribución de probabilidad} sobre \( A \) es una función \( \mu : A
\rightarrow [0, 1] \) tal que \( \sum_{a \in A} \mu(a) = 1 \).

Una distribución \( \mu \) es \textbf{racional} si \( \mu(a) \) es racional
para cada \( a \in A \), positiva si \( \mu(a) > 0 \) para cada \( a \in A \),
y \textbf{de Dirac} si \( \mu(a) = 1 \) para algún \( a \in A \). Una
distribución de Dirac \( \mu \) donde \( \mu(a) = 1 \) también se denota por \(
\mu_a \) o simplemente \( a \).

Sea \( \Omega \) un conjunto de eventos elementales. Una
\textbf{\(\sigma\)-álgebra} sobre \( \Omega \) es un conjunto \( \mathcal{F}
\subseteq 2^{\Omega} \) que incluye \( \Omega \) y está cerrado bajo
complementos y uniones numerables. Un \textbf{espacio medible} es un par \(
(\Omega, \mathcal{F}) \) donde \( \mathcal{F} \) es una \(\sigma\)-álgebra
sobre \( \Omega \). Una función \textbf{\( \mathcal{F} \)-medible} sobre \(
(\Omega, \mathcal{F}) \) es una función \( f : \Omega \rightarrow \mathbb{R} \)
tal que \( f^{-1}(I) \in \mathcal{F} \) para cada intervalo \( I \) en \(
\mathbb{R} \).

Una \textbf{medida de probabilidad} sobre un espacio medible \((\Omega,
\mathcal{F})\) es una función \( \mathcal{P} : \mathcal{F} \rightarrow [0, 1]
\) tal que, para cada colección numerable \(\{A_i\}_{i \in I}\) de elementos
disjuntos de a pares de \(\mathcal{F}\), tenemos que \( \mathcal{P}\left(
\bigcup_{i \in I} A_i \right) = \sum_{i \in I} \mathcal{P}(A_i) \), y además \(
\mathcal{P}(\Omega) = 1 \).

Un \textbf{espacio de probabilidad} es una terna donde \((\Omega,
\mathcal{F})\) es un espacio medible y \(\mathcal{P}\) es una medida de
probabilidad sobre \((\Omega, \mathcal{F})\).

Una \textbf{variable aleatoria} sobre un espacio de probabilidad \((\Omega,
\mathcal{F}, \mathcal{P})\) es una función \(\mathcal{F}\)-medible \(X : \Omega
\rightarrow \mathbb{R}\). El \textbf{valor esperado} de \(X\), denotado por
\(E(X)\), se define como la integral de Lebesgue \(\int_{\Omega} X \,
d\mathcal{P}\).

Una \textbf{variable aleatoria \(X\) es discreta} si existe un subconjunto
finito o numerablemente infinito \(N\) de \(\mathbb{R}\) tal que \(P(X^{-1}(N))
= 1\). El valor esperado de una variable aleatoria discreta \(X\) es igual a
\(\sum_{n \in N} n \cdot \mathcal{P}(X=n)\), donde \(X=n\) denota el conjunto
\(X^{-1}(\{n\})\).

\textbf{$\sigma$-álgebra de Borel}

Sea \( T = (S, \rightarrow) \) un sistema de transiciones. Sea \( \mathcal{B}
\) la menor \(\sigma\)-álgebra sobre \( \textit{Run} \) que contiene todos los
cilindros básicos \( \textit{Run}(w) \) donde \( w \in \textit{Fpath} \) (es
decir, \( \mathcal{B} \) es la \(\sigma\)-álgebra de Borel generada por
conjuntos abiertos en la topología de Cantor sobre \( \textit{Run} \)).
Entonces, \((\textit{Run}, \mathcal{B})\) es un espacio medible, y los
elementos de \( \mathcal{B} \) se llaman \textbf{conjuntos de Borel de
	ejecuciones}.

La \textbf{\(\sigma\)-álgebra de Borel} \( \mathcal{B} \) contiene muchos
elementos interesantes. Por ejemplo, sea \( s, t \in S \) y sea \(
\textit{Reach}(s, t) \) el conjunto de todas las ejecuciones iniciadas en \( s
\) que visitan \( t \). Obviamente, \( \textit{Reach}(s, t) \) es la unión de
todos los cilindros básicos \( \mathcal{R}\text{un}(w) \) donde \( w \in
\text{Fpath}(s) \) y \( w \) visita \( t \), y por lo tanto \( \text{Reach}(s,
t) \in \mathcal{B} \).

De manera similar, se puede demostrar que el conjunto de todas las ejecuciones
iniciadas en \( s \) que visitan \( t \) infinitamente a menudo es Borel. En
realidad, la mayoría de los conjuntos `interesantes' de ejecuciones son de
Borel, aunque también existen subconjuntos de \(\textit{Run}\) que no están en
\(\mathcal{B}\).

Sea \( A \in \mathcal{B} \), y sea \( f : \textit{Run} \rightarrow \{0, 1\} \)
una función que asigna a un dado \( w \in \mathcal{R}\text{un} \) 1 o 0,
dependiendo de si \( w \in A \) o no, respectivamente. Entonces, \( f \) es
\(\mathcal{B}\)-medible, porque para cada intervalo \( I \) en \( \mathbb{R} \)
tenemos que \( f^{-1}(I) \) es igual a \(\mathcal{R}\text{un}\), \( A \),
\(\mathcal{R}\text{un} \setminus A \) o \( \emptyset \), dependiendo de si \( I
\cap \{0, 1\} \) es igual a \{0, 1\}, \{1\}, \{0\}, o \(\emptyset\),
respectivamente.

\section{Rabin}

