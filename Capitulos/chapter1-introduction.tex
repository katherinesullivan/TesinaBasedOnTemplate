\chapter{Introducción}
~\label{cap:intro}

Ejemplo de introducción:

Discutimos todo el tiempo si $P = NP$~\cite{p-np}. Este problema es interesante
porque \ldots (es divertido?). Mi aporte es que logro resolver esta duda.
Finalmente, lograremos demostrar que el Algoritmo~\ref{alg:1} que es P, es
igual al algoritmo~\ref{alg:2} que es NP\@. Gracias.

\section{Contribuciones}
~\label{cap:intro:sec:contributions}

En esta tesina lograré probar que $P = NP$ mediante \ldots

\section{Organización del trabajo}
~\label{cap:intro:sec:outline}

\begin{itemize}
	\item En el \textbf{Capítulo~\ref{cap:art}}, presentaré \ldots ;

	\item En el \textbf{Capítulo~\ref{cap:approach}}, extenderé y explicaré en detalle
	      los métodos de esta tesina \ldots ;

	\item En el \textbf{Capítulo~\ref{cap:results}}, presentaré mis resultados y
	      conseguidos \ldots;

	\item Finalmente, en el \textbf{Capítulo~\ref{cap:conclusions}}, concluyo con un
	      resumen de los aportes realizados en esta tesina, menciono las implicancias de
	      esta investigación y hallazgos realizados, y sugiero potenciales caminos para
	      futuras investigaciones. % Quote directa de mi tesina.

\end{itemize}

