\chapter{Introducción}
~\label{cap:intro}

A lo largo de la historia, la teoría de juegos ha cumplido un rol central en
muchas áreas de las ciencias de la computación. En particular, la aparición del
concepto de \textit{juego estocástico} ha tenido un impacto significativo en la
modelización y estudio de sistemas reactivos, sistemas con comportamientos no
deterministas, y sistemas con incertidumbre.

El concepto de juego estocástico se contrapone al de juego determinista.
Mientras que en el último cada acción de un jugador resulta en un único próximo
estado del juego, en el primero cada acción de un jugador determina una
distribución de probabilidad sobre los posibles próximos estados del juego. Es
por esto que los juegos estocásticos han resultado una herramienta útil para
modelar sistemas en diversas áreas como finanzas, inteligencia artificial y
telecomunicaciones.

Existen sistemas donde las posibles transiciones de un estado a otro no son
discretas, sino que existen una cantidad continua de opciones, y estas se
suelen poder describir a partir de un conjunto de restricciones. Ejemplos de
estos sistemas son redes de comunicaciones, modelos financieros complejos y
planificaciones de movimiento en robótica. Para modelar este tipo de
situaciones surgió el concepto de juegos estocásticos politópicos. En ellos,
los posibles siguientes estados de un juego forman diferentes politopos en el
espacio de estados, y la decisión de un jugador lo que determinará será un
politopo y una distribución de probabilidad sobre ese politopo, a partir de los
cuales se verá cuál es el próximo estado del juego.

Para su estudio, y en general para el estudio de juegos, la pregunta central a
responder es si un jugador tiene una estrategia ganadora para el juego. Esto
dependerá de cómo se define el ganar para el jugador. En juegos estocásticos,
una manera definir el ganar es a través de \textit{objetivos} que se dan en
forma de una \textit{propiedad $\omega$-regular} $\phi$. Como nos
concentraremos en juegos de suma cero, el ganar para un jugador será que se
cumpla $\phi$, mientras que su contrincante ganará si se cumple $\neg \phi$.

Dentro de las propiedades $\omega$-regulares se destacan \textit{las
	condiciones de Rabin} y su dual, \textit{las condiciones de Street}. La
relevancia práctica de ellas viene del hecho que su forma se corresponde con
aquella de las condiciones de equidad en sistemas de transición.

\section{Fundamentos y estado del arte}

Los juegos estocásticos fueron introducidos por L. Shapley en 1953
\cite{Shapley1953} con el fin de modelar interacciones dinámicas donde el
entorno cambia en respuesta a los comportamientos de los jugadores. Estos
extienden el modelo de los procesos de decisión de Markov, desarrollado por
varios investigadores en la RAND Corporation en el período de 1949 a 1952, a
situaciones competitivas con más de un tomador de decisiones.

Gracias a su aporte para la modelización en varias áreas, resultó lógica la
división de juegos estocásticos en base a distintos tipos de funciones de pago.
El estudio de la existencia, tipo y computabilidad de las estrategias óptimas
suele considerarse en juegos con funciones de recompensa cuantitativas (eg.,
funciones de recompensa media, descontada, pesada o límite) y cualitativas
(eg., funciones de recompensas asociadas a objetivos de alcanzabilidad, de
Büchi, de Rabin o de Müller) \cite{Chatterjee2007,Kučera2011,Chatterjee1}.

Resultados para juegos estocásticos con funciones de recompensa media y
descontada fueron reportados desde 1958 por Gillete \cite{Gillette1958},
mientras que los primeros resultados para juegos con objetivos de
alcanzabilidad fueron publicados por Condon en 1992 \cite{CONDON1992} y los
juegos con objetivos de Rabin comenzaron a ser estudidados en profundidad por
Chaterjee desde 2005 \cite{ComplexityRabin,Chatterjee2007,Chatterjee1}.

Recientemente, en 2024, Castro y D'Argenio introdujeron el concepto de juegos
estocásticos politópicos \cite{Polytopal}, los cuales expanden la noción de
juegos estocásticos.
%con incertidumbre sobre las probabilidades de las acciones de los jugadores. 
En este mismo paper, se presentan resultados para juegos estocásticos
politópicos con objetivos de alcanzabilidad, funciones de pago descontadas y
funciones de pago promedio.

\section{Contribuciones}
~\label{cap:intro:sec:contributions}

La idea de este trabajo será extender el estudio de juegos estocásticos
politópicos a juegos con objetivos de Rabin.

\section{Organización del trabajo}
~\label{cap:intro:sec:outline}

\begin{itemize}
	\item En el \textbf{Capítulo~\ref{cap:art}}, presentaré \ldots ;

	\item En el \textbf{Capítulo~\ref{cap:approach}}, extenderé y explicaré en detalle
	      los métodos de esta tesina \ldots ;

	\item En el \textbf{Capítulo~\ref{cap:results}}, presentaré mis resultados y
	      conseguidos \ldots;

	\item Finalmente, en el \textbf{Capítulo~\ref{cap:conclusions}}, concluyo con un
	      resumen de los aportes realizados en esta tesina, menciono las implicancias de
	      esta investigación y hallazgos realizados, y sugiero potenciales caminos para
	      futuras investigaciones. % Quote directa de mi tesina.

\end{itemize}

