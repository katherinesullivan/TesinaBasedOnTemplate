\chapter{Introducción}
~\label{cap:intro}
\vspace{-1cm}

A lo largo de la historia, la teoría de juegos ha cumplido un rol central en
muchas áreas de las ciencias de la computación. En particular, la aparición del
concepto de \textit{juego estocástico} ha tenido un impacto significativo en la
modelización y estudio de sistemas reactivos, sistemas con comportamientos no
deterministas, y sistemas con incertidumbre.

El concepto de juego estocástico se contrapone al de juego determinista.
Mientras que en el último cada acción de un jugador resulta en un único próximo
estado del juego, en el primero cada acción de un jugador determina una
distribución de probabilidad sobre los posibles próximos estados del juego. Es
por esto que los juegos estocásticos han resultado una herramienta útil para
modelar sistemas en diversas áreas como finanzas, inteligencia artificial y
telecomunicaciones.

%Rechequear este parrafo
Existen sistemas en donde las probabilidades de transición de un estado a otro
no son conocidas con exactitud, sino que se sabe un conjunto de restricciones
sobre las mismas, y en ese caso podríamos tener una cantidad infinita de
posibilidades que cumplan esas restricciones. Ejemplos de estos sistemas son
redes de comunicaciones, modelos financieros complejos y planificaciones de
movimiento en robótica. Para modelar este tipo de situaciones surgió el
concepto de juegos estocásticos politópicos. En ellos, los posibles siguientes
estados de un juego forman diferentes politopos en el espacio de estados, y la
decisión de un jugador lo que implicará la elección de un politopo y una
distribución de probabilidad sobre ese politopo, a partir de los cuales se verá
cuál es el próximo estado del juego.

Para su estudio, y en general para el estudio de juegos, la pregunta central a
responder es si un jugador tiene una estrategia ganadora. Esto dependerá de
cómo se define el ganar para el jugador. En juegos estocásticos, una manera de
definir el ganar es a través de \textit{objetivos} dados en forma de una
\textit{propiedad $\omega$-regular} $\phi$. Como nos concentraremos en juegos
de suma cero, el ganar para un jugador será que se cumpla $\phi$, mientras que
su contrincante ganará si se cumple $\neg \phi$.

Dentro de las propiedades $\omega$-regulares se destacan \textit{las
	condiciones de Rabin} y su dual, \textit{las condiciones de Street}. La
relevancia práctica de ellas viene del hecho de que su forma se corresponde con
aquella de las condiciones de equidad en sistemas de transición, además de que
cualquier propiedad $\omega$-regular puede ser reescrita como una propiedad de
Rabin \cite{AutomataLogicsInfiniteGames}.

Una vez fijado un jugador (generalmente el jugador ``maximizador''/ el jugador
que busca que se cumpla la propiedad $\phi$), al problema de la computación de
los estados estados ganadores de un juego se lo suele llamar el análisis
cualitativo del juego, mientras que el análisis cuantitativo del juego refiere
al problema de computar, para cada estado, cuál es la probabilidad máxima que
tiene este de ganar, aún cuando esta sea menor que 1.

En el trabajo nos concentraremos en el análisis cualitativo de juegos
estocásticos politópicos con objetivos de Rabin. Mostraremos que el conjunto de
sus estados ganadores son iguales a los estados ganadores de un particular
juego determinista construido a partir del juego estocástico, y presentaremos
un algoritmo para su cómputo y analizaremos su complejidad.

%Esta igualdad que mostramos también permite que cualquier avance en términos de mejoras para el cómputo de conjuntos de estados ganadores en juegos deterministas, resultará en una mejora para el cómputo de conjuntos de estados ganadores en juego estocásticos politóìcos con objetivos de Rabin.

\section{Estado del arte}

Los juegos estocásticos fueron introducidos por L. Shapley en 1953
\cite{Shapley1953} con el fin de modelar interacciones dinámicas donde el
entorno cambia en respuesta a los comportamientos de los jugadores. Estos
extienden el modelo de los procesos de decisión de Markov, desarrollado por
varios investigadores en la RAND Corporation en el período de 1949 a 1952, a
situaciones competitivas con más de un tomador de decisiones.

% El problema de síntesis (o problema de control) para los sistemas reactivos refiere a la construcción de una estrategia ganadora en el correspondiente juego. Este problema fue planteado por primera vez por Alonzo Church \cite{28, 8} y Richard Büchi en condiciones que pueden ser reducidads a juegos deterministas con objetivos $\omega$-regulares. Estos problemas fueron resueltos por .... Ahora bien, la restricción a juegos deterministas resultaba restrictiva, pues pueden existir situaciones en las que es útil modelar comportamiento que no es estrictamente adversarial \cite{88,31}. Es por esto que el estudio con esta formulación fue prosperando en los años posteriores.

Gracias a su aporte para la modelización en varias áreas, resultó lógica la
división de juegos estocásticos en base a distintos tipos de funciones de pago.
El estudio de la existencia, tipo y computabilidad de las estrategias óptimas
suele considerarse en juegos con funciones de recompensa cuantitativas (eg.,
funciones de recompensa media, descontada, pesada o límite) y cualitativas
(eg., funciones de recompensas asociadas a objetivos de alcanzabilidad, de
Büchi, de Rabin o de Müller) \cite{Chatterjee2007,Kučera2011,Chatterjee1}.

Resultados para juegos estocásticos con funciones de recompensa media y
descontada fueron reportados desde 1958 por Gillete \cite{Gillette1958},
mientras que los primeros resultados para juegos con objetivos de
alcanzabilidad fueron publicados por Condon en 1992 \cite{CONDON1992} y los
juegos con objetivos de Rabin comenzaron a ser estudidados en profundidad por
Chaterjee desde 2005 \cite{ComplexityRabin,Chatterjee2007,Chatterjee1}.

Recientemente, en 2024, Castro y D'Argenio introdujeron el concepto de juegos
estocásticos politópicos \cite{Polytopal}, los cuales expanden la noción de
juegos estocásticos.
%con incertidumbre sobre las probabilidades de las acciones de los jugadores. 
En este mismo paper, se presentan resultados para juegos estocásticos
politópicos con objetivos de alcanzabilidad, funciones de pago descontadas y
funciones de pago promedio. Nuestro objetivo aquí será extender el estudio de
los juegos estocásticos politópicos con objetivos de Rabin.

\section{Organización del trabajo}
~\label{cap:intro:sec:outline}
\begin{itemize}
	\item En el \textbf{Capítulo~\ref{cap:modelos}}, nos adentraremos en el campo de la
	      verificación de modelos con juegos, presentando definiciones básicas de modelos
	      y juegos acompañadas con ejemplos que servirán de guía para el entendimiento de
	      los mismos;

	\item En el \textbf{Capítulo~\ref{cap:objetivos}}, presentaremos en profundidad los
	      objetivos $\omega$-regulares haciendo especial hincapié en los objetivos de
	      Rabin;

	\item En el \textbf{Capítulo~\ref{cap:psg}}, veremos definiciones formales referidas
	      a los juegos estocásticos politópicos y los resultados obtenidos en el paper
	      \textit{Polytopal Stochastic Games} \cite{Polytopal};

	\item En el \textbf{Capítulo~\ref{cap:results}}, desarrollaremos los resultados
	      originales de esta tesina, presentando en la sección~\ref{sec:pmdp} a los
	      procesos de decisión de Markov politópicos con sus teoremas asociados y
	      mostrando en la sección~\ref{sec:prueba} cómo calcular el conjunto de estados
	      ganadores con probabilidad 1 para el jugador maximizador en un juego
	      estocástico politópico con objetivo de Rabin, y

	\item Finalmente, en el \textbf{Capítulo~\ref{cap:conclusions}}, concluiremos con un
	      resumen de los aportes realizados en esta tesina y nos expandiremos sobre
	      potenciales caminos para futuras investigaciones en el tema.

\end{itemize}

