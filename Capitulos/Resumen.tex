\chapter*{Resumen}
~\label{abstract}
\addcontentsline{toc}{chapter}{Resumen}
\vspace{-1cm}

Los juegos estocásticos han servido para modelar diversos sistemas del mundo
real donde se aprecia un comportamiento estocástico y adversarial. Ejemplos de
esto pueden ser ,,, ,,,.

En 2025, Castro y D'Argenio publican el paper \emph{Polytopal Stochastic
	Games}, en donde por primera vez se presenta el concepto de juego estocástico
politópico, respondiendo a la necesidad de modelar juegos estocásticos que
puedan capturar mayor incertidumbre sobre las distribuciones de probabilidad
que determinan las acciones que toman los distintos jugadores. En el paper
estos son estudiados en relación a funciones de recompensa y objetivos de
alcanzabilidad.

Este trabajo se concentrará en extender el estudio de juegos estocásticos
politópicos con objetivos de Rabin. Los objetivos de Rabin permiten describir
especificaciones sobre conjuntos de estados que deben ser visitados infinitas
veces y conjuntos de estados que deben ser visitados una cantidad finita de
veces. A lo largo del trabajo nos adentraremos sobre por qué resulta
interesante estudiar juegos estocásticos politópicos, por qué resulta de
importancia estudiar objetivos de Rabin y veremos cómo dar respuestas a las
preguntas de quién gana y de cómo se gana un juego estocástico politópico con
objetivo de Rabin.