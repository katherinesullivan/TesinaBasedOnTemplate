\chapter{Objetivos de Rabin en PSGs}
~\label{cap:results}

\section{Procesos de Decisión de Markov Politópicos - PMDPs}
\label{sec:pmdp}

\hl{Acá se está dando directamente la interpretación de un PMDP}

Lo podría definir directamente como PSGs con un solo jugador.

\begin{definition}
	Un \textit{proceso de decisión de Markov politópico} (PMDP por sus siglas en inglés) es una tupla $\M = (\St, Act, \theta')$ donde $S$ es un conjunto finito de estados, $Act$ es un conjunto de pares (politopo, distribución) y $\theta': \St \times Act \times \St \rightarrow [0, 1]$ es la función de transición entre estados. También podemos definir a un PDMP como un PSG donde $\St_\cuad = \emptyset$ o $\St_\diam = \emptyset$.
\end{definition}

% También introducimos el concepto de \textit{comportamiento} en un PMDP. Un
% comportamiento será una secuencia alternante de estados y conjuntos de próximos
% estados posibles, que refleja un proceso de selección de dos pasos. Desde un
% estado s, primero, el jugador selecciona un politopo y una distribución cuyo
% conjunto soporte será un $V \in V_s$, y luego el próximo estado $s'$ será
% elegido probabilísticamente en base a la distribución seleccionada. Presentamos
% la definición formal de la siguiente manera:

Los caminos, las estrategias, y la medida de probabilidad en los PMDPs se
definen tomando las mismas definiciones de los juegos estocásticos politópicos.

A continuación presentaremos las definiciones de conjuntos estado-resultado,
sub-PMDPs y componentes finales que en cierta forma extienden a los conceptos
de conjuntos estado-acción, sub-MDPs y componentes finales que se suelen
encontrar en la literatura de procesos de decisión de Markov
\cite{AlfaroThesis,BaierKatoen}.

\begin{definition}[Conjuntos estado-resultado y sub-PMDPs]
	Dado un PMDP $\M = (S, Act, \theta')$ un conjunto estado-resultado es un subconjunto $\chi \subseteq \{(s, V') \mid s \in S \wedge V' \in V_s\}$. Un sub-PMDP es un par $(C, D)$, donde $C \subseteq S$ y $D$ es una función que asocia a cada $s \in C$ un conjunto $D(s) \subseteq V_s$ de subconjuntos de estados próximos posibles. Hay una relación uno-a-uno entre sub-PMDPs y conjuntos de estado-acción:

	\begin{itemize}
		\item dado un conjunto estado-resultado $\chi$, denotamos $\sub(\chi) = (C, D)$ al
		      sub-PMDP definido por:
		      \[
			      C = \{s \mid \exists V' . (s, V') \in \chi\} \quad D(s) = \{V' \mid (s, V') \in \chi\}
		      \]

		\item dado un sub-PMDP $(C, D)$, denotamos por $\er(C,D) = \{(s,V') \mid s \in C
			      \wedge V' \in D(s)\}$ al conjunto estado-resultado correspondiente a $(C, D)$.
	\end{itemize}
\end{definition}

Si definimos para cada $V'$, un vértice único nuevo $v_{V'}$ podemos ver que
cada sub-PMDP $(C, D)$ induce una \textit{relación de aristas}: hay una arista
$(s, v_{V'})$ de $s \in C$ a $v_{V'}$ para cada $V' \in D(s)$ y hay una arista
de $(v_{V'}, t)$ de $v_{V'}$ con $V' \in D(s)$ a $t \in \St$ sii es posible ir
de $s$ a $t$ en un paso con probabilidad positiva utilizando una acción cuyo
conjunto resultado sea $V'$. La definición formal es como sigue:

\begin{definition}[Relación de aristas $\rho$]
	Para un sub-PMDP, definimos la relación $\rho_{(C,D)}$ como
	\[
		\rho_{(C,D)} = \{(s, v_{V'}) \mid \exists V' \in D(s)\} \cup \{(v_{V'}, t) \mid t \in V' \}
	\]
\end{definition}

\begin{definition}[Componente final] \label{defEC}
	Un sub-PMDP es una componente final si:

	\begin{itemize}
		\item $V' \subseteq C$ para todo $V'$ tal que existe un $s \in C$ donde $V' \in D(s)$

		\item el grafo $(C \cup \{v_{V'} \mid \exists s \in C : V' \in D(s)\}, \rho_{(C,D)})$
		      es fuertemente conexo.
	\end{itemize}

	%Decimos que una componente final está contenida en un sub-PMDP $(C', D')$ si $er(C, D) \subseteq er(C', D')$; decimos que una componente final $(C, D)$ es maximal en un sub-PMDP $(C', D')$ si no hay ninguna otra componente final $(C'', D'')$ tal que $er(C,D) \subset er(C'', D'') \subseteq (C,D)$.
	Llamaremos $\EC(\M)$ al conjunto de todas las componentes finales en un PMDP
	$\M$.
\end{definition}

Intuitivamente, una componente final representa un conjunto de pares
estado-resultado que, una vez en ellos, es posible quedase allí para siempre si
la estrategia escoge las acciones de manera apropiada. Esta intuición se hará
precisa con los siguientes teoremas.

Antes de enunciar estos teoremas, introducimos una abreviatura para el conjunto
de estados-resultados que ocurren infinitamente a menudo en un camino dado. %y una notación para el conjunto (infinito) de acciones con el mismo conjunto resultado.

\begin{definition}[$\inft$]
	Dado un camino $\omega = (s_0, \alpha_0, s_1, \alpha_1, \dots)$ indicamos por
	\[
		\inft (\omega) = \{(s, V') \mid s_k = s \wedge supp(\alpha_k) = V' \text{ para infinitos } k \in \mathbb{N}_0\}
	\]
	al conjunto de pares estado-resultado que ocurren infinitas veces en él.
\end{definition}

Ahora sí, podemos pasar a presentar las primeras demostraciones:

\begin{theorem}[Estabilidad de componentes finales]
	\label{teoEstabilidadEC}
	Sea $(C, D)$ una componente final. Entonces, para cada estrategia $\pi$ existe una estrategia $\pi'$, que difiere de $\pi$ solo en $C$, tal que:
	\begin{equation} \label{estabEC}
		\Prob^\pi_s(\diam C) = \Prob^{\pi'}_s(\diam C) = \Prob^{\pi'}_s(\inft(\omega) = \er(C,D))
	\end{equation}
	para todo $s \in S$.
\end{theorem}

\begin{proof}
	Considérese una estrategia $\pi'$ definida como sigue para cada secuencia $s_0 \dots s_n$ con $n \geq 0$:

	\begin{itemize}
		\item Si $s_n \in C$, la estrategia asignará probabilidad positiva a una única acción
		      $(K, \mu) \in A(s_n, V')$ para cada conjunto resultado $V'$ (notaremos a esta
		      acción particular con $(K, \mu)_{V'}$), y la probabilidad de elegir cada una de
		      esas acciones se distribuirá de manera uniforme. Es decir,
		      \[
			      \pi'(s_0 \dots s_n)(K, \mu) =
			      \begin{cases}
				      \frac{1}{\abs{D(s_n)}} \quad \text{si } (K, \mu) = (K, \mu)_{V'} \text{ para algún } V' \in D(s); \\
				      0 \qquad \quad \text{en otro caso}
			      \end{cases}
		      \]

		\item Si $s_n \notin C$, la estrategia $\pi'$ coincide con $\pi$, i.e.
		      \[
			      \pi(s_0 \dots s_n)(K, \mu) = \pi'(s_0\dots s_n)(K, \mu) \quad \forall (K, \mu) \in Act
		      \]

		      %debería ser \pi(s_0 \dots s_n)
	\end{itemize}

	La primera igualdad en \ref{estabEC} es una consecuencia del hecho de que $\pi$
	y $\pi'$ coinciden fuera de $C$.

	Para la segunda igualdad, basta con ver que bajo la estrategia $\pi'$ una vez
	que un camino entra a $C$, nunca sale de $C$ ni se elige una acción que no esté
	en $D$. Es más, una vez en $C$ un camino visitará todos los estados de $C$
	infinitamente a menudo con probabilidad 1.
\end{proof}

Este teorema nos permite presentar un corolario que nos será útil al momento de
probar el teorema principal de esta tesina.

\begin{corollary}
	\label{adaptB30}
	Sea $\M$ un PMDP y sea $s$ un estado ganador en él para una condición de Rabin $R = \{(E_1, F_1), \dots, (E_d, F_d)\}$.%, cuya especificación LTL es $\varphi$ (es decir, para toda estrategia $\pi$, $\ProbPMDP(\varphi) = 1$). 
	Entonces, para toda componente final $(C,D)$ alcanzable desde $s$, vale que existe algún $j \in [1, d]$ tal que $C \cap E_j= \emptyset$ y $C \cap F_j \neq \emptyset$.
\end{corollary}

\begin{proof}
	Supongamos que existe una componente final $(C', D')$ alcanzable desde $s$ tal que esta no cumple que existe algún $j \in [1, d]$ tal que $C' \cap E_j= \emptyset$ y $C' \cap F_j \neq \emptyset$.

	Por teorema~\ref{estabEC}, sabemos que existe una estrategia válida $\pi'$ tal
	que permite llegar a $C'$ desde $s$ y quedarse allí para siempre. Podemos
	plantear un camino que sigue esa estrategia $\pi'$, llamemóslo $\omega'$. Ahora
	bien, entonces también por como presentamos que será esa estrategia $\pi'$ en
	el teorema~\ref{estabEC}, sabemos que $\Inf(\omega') = C$. Pero, a su vez
	sabemos que ese camino debe cumplir con la condición de Rabin, es decir,
	sabemos que existe algún $j \in [1, d]$ tal que $\Inf(\omega') \cap E_j =
		\emptyset$ y $\Inf(\omega') \cap F_j \neq \emptyset$, lo que quiere decir que
	existe algún $j \in [1, d]$ tal que $C' \cap E_j= \emptyset$ y $C' \cap F_j
		\neq \emptyset$ y tendríamos una contradicción.

	Esta contradicción viene de suponer que existe tal comopnente final. Con lo que
	probamos lo que queríamos.
\end{proof}

El próximo resultado establece que, para cualquier estado inicial y cualquier
estrategia (de memoria finita), un camino terminará con probabilidad 1 en una
componente final. Esta es la razón detrás del nombre ``componente final".

\begin{theorem}[teorema fundamental de las componentes finales] \label{teoFundEC}
	Sea $\M$ un PMDP. Para todo $s \in S$, toda estrategia $\pi$ de memoria finita,
	$$\Prob^\pi_{\M, s}(\{ \omega \in \Paths(s) \mid \sub(\inft(\omega)) \text{ es una componente final}\}) = 1$$
\end{theorem}

\begin{proof}
	Consideremos un sub-PMDP $(C,D)$ que no sea una componente final y sea $\Paths_s^{(C,D)} = \{\hat{\omega} \in \Paths_s \mid \inft(\hat{\omega}) = \er(C, D)\}$ el conjunto de caminos cuyo conjunto de pares estado-resultado que se repiten infinitas veces en él forman el sub-PMDP $(C,D)$.
	%(otra forma de decirlo capaz podría ser: conjunto de caminos que se estabilizan en el sub-PMDP $(C,D)$).

	Si podemos mostrar que
	\begin{equation}\label{omegaEnOmega0}
		\Prob^\pi_{\M, s}(\{ \omega \in \Paths(s) \mid \omega \in \Paths_s^{(C,D)}\}) = 0
	\end{equation}
	como $(C,D)$ es un sub-PMDP cualquiera y como hay una cantidad finita de sub-PMDPs en $\M$, esto es lo mismo que mostrar que
	$$\Prob^\pi_{\M, s}(\{ \omega \in \Paths(s) \mid \sub(\inft(\omega)) \text{ es una componente final}\}) = 1$$.
	Veamos que vale \ref{omegaEnOmega0}, dividiendo en casos según cuál es la
	condición de la definición \ref{defEC} que no se cumple para $(C,D)$:

	\begin{itemize}
		\item Primero, asumamos que existe un $(t, V') \in \er(C,D)$ tal que $V' \nsubseteq
			      C$.

		      %Vamos a usar un argumento similar al de la prueba de la primera contención en el intento anterior.

		      %Para cada $(K, \mu) \in A(t, V')$ podemos definir $r_{(K, \mu)} = \sum_{u \in C} \mu(u)$, la probabilidad de quedarnos en $C$ eligiendo la acción $(K, \mu)$, y sabemos que $r_{(K, \mu)} < 1$. 

		      Sabemos que cada camino en $\Paths_s^{(C,D)}$ toma el par estado-resultado $(t,
			      V')$ infinitas veces. Llamemos $I$ al conjunto de índices infinito que
		      representa los momentos en los que se visita $(t, V')$. Indiquemos con $\mu_i$
		      a la distribución elegida en el momento $i \in I$ y definamos como $r_i =
			      \sum_{u \in C} \mu_i(u)$ a la probabilidad de quedarnos en $C$ en el momento $i
			      \in I$ (que en cada caso será menor a 1 porque $V' \nsubseteq C$). %y cada $\mu$ tal que existe un par $(K, \mu)$ en $A(t, V')$ $\mu$ asigna probabilidad positiva a cada $v \in V'$). %y llamemos al evento de q

		      Como $\pi$ es de memoria finita sucederá que $\pi$ solo puede elegir una
		      cantidad finita de acciones (y, por lo tanto, distribuciones) distintas desde
		      $t$. Esto hace que el conjunto $R = \{r_i \mid i \in I\}$ tenga un máximo,
		      llamémoslo $r$.

		      Para que valga que $\omega$ esté en $\Paths_s^{(C,D)}$ tiene que valer que en
		      infinitos momentos $i$ nos quedemos en $C$. Entonces que vale que $\Prob_s^\pi
			      (\omega \in \Paths_s^{(C,D)}) < r^k$ para todo $k > 0$ natural. Como sabemos
		      que $r < 1$, tenemos que $\Prob^\pi_{\M, s}(\{ \omega \in \Paths(s) \mid \omega
			      \in \Paths_s^{(C,D)}\}) = 0$.

		      %Es decir para un subconjunto infinito $J \subseteq I$, vale que: $$\Prob_{s}^{\pi}(\omega \in \Omega_s^{(C,D)}) \leq \prod_{i \in J} r_i$$

		      %Es decir que queremos ver que, si llamamos $C_n$ al evento de quedarnos en $C$ en el momento $n$,

		      %$\Prob(\limsup_{n \to \infty} C_n) = 1$. Por el segundo teorema de Borel-Cantelli sabemos que esto pasa si $\sum_{n>1} \Prob(C_n)$ diverge.
		      %Podemos probar que $\sum_{n>1}\Prob(C_n)$ diverge porque existe un 

		      %Entonces, tenemos que $\Prob_s^{\pi}(\omega \in \Omega_s^{(C,D)}) = \Pi_{i \in I} $

		      %$DistInf \subseteq A(t, V')$ al conjunto de pares $(K, \mu)$ que se toman en los infinitos momentos que se visita $(t, V')$. Entonces, tenemos que         $$\Prob_s^{\pi}(\omega \in \Omega_s^{C,D}) < \Pi_{(K, \mu) \in DistInf} r_{(K, \mu)}$$ Como $\pi$ es de memoria finita, existirá un conjunto infinito $Ig \subseteq DistInf$

		      %Como cada camino en $\Omega_s^{(C,D)}$ toma el par estado-resultado $(t, V')$ infinitas veces, tenemos que $\Prob_s^{\pi}(\omega \in \Omega_s^{C,D}) < \Pi_{(K, \mu) \in A(t, V')} r_{(K, \mu)}$ para todo $k \in \mathbb{N}$ y como $r < 1$, tenemos que 

		      %(No podemos decir que existe un máximo $r_{(K, \mu)}$ porque $A(s, V')$ no es un conjunto finito (ni tampoco que existe una cota superior) así que no sabría si está bien / si lo puedo formalizar más / si puedo definir algún $r$)

		\item Si no, asumamos que existen $t_1, t_2 \in C$ tales que no hay camino de $t_1$ a
		      $t_2$ en $(C \cup \{v_V' \mid \exists s \in C : V' \in D(s)\}, \rho(C,D))$.

		      La falta de camino de $t_1$ a $t_2$ en $(C, \rho_{(C,D)})$ implica que para
		      cada subsecuencia $s_m V_m s_{m+1} ... s_n$ de camino en $\Paths_s^{(C,D)}$ que
		      vaya de $s_m=t_1$ a $s_n=t_2$, hay 2 opciones:

		      \begin{enumerate}
			      \item existe un $j \in [m+1, n-1]$ tal que $s_j \notin C$. Como cada camino en
			            $\Paths_s^{(C,D)}$ contiene infinitas subsecuencias de $t_1$ a $t_2$ y tenemos
			            una cantidad finita de estados, sabemos que habrá una cantidad infinita de
			            $s_j$ iguales. Pero si infinitas veces se toma un estado $s_j$ entonces, $s_j
				            \in C$, lo que contradice la hipótesis anterior. Absurdo.

			      \item existe un $j \in [m, n-1]$ tal que $V_j \notin D(j)$. Como cada camino en
			            $\Paths_s^{(C,D)}$ contiene infinitas subsecuencias de $t_1$ a $t_2$ y tenemos
			            una cantidad finita de conjuntos resultado, sabemos que habrá una cantidad
			            infinita de $V_j$ iguales, con lo que $V_j \in D(j)$, lo que contradice la
			            hipótesis anterior. Absurdo
		      \end{enumerate}
		      Con lo que arribamos a que $\Prob^\pi_{\M, s}(\{ \omega \in \Paths(s) \mid \omega \in \Paths_s^{(C,D)}\}) = 0  $
	\end{itemize}

\end{proof}

Esto nos permite definir el siguiente corolario útil y clásico en los procesos
de decisión de Markov para el análisis de condiciones de Rabin.

\begin{corollary}
	Sea $\M$ un PMDP, $s$ un estado en él y sea $\pi$ una estrategia de memoria finita en él. Una condición de Rabin $R = \{ (E_1, F_1), \dots, (E_d, F_d)\}$ se satisface desde $s$, siguiendo la estrategia $\pi$ con probabilidad 1 si y solo si para cada componente final $U$ alcanzable desde $s$, existe un $j \in \{ 1, \dots, d\}$ tal que $U \cap E_j = \emptyset$ y $U \cap F_j \neq \emptyset$.
\end{corollary}

Ahora bien, una pregunta muy natural que surge es "¿por qué nos restringimos a
estrategias de memoria finita en este último teorema?"

\textbf{Nota sobre la limitación a estrategias finitas del Teorema~\ref{teoFundEC}}

Claro, en un principio la idea fue poder probar ambos teoremas en su caso
general, sin tener que limitarnos en la clase de las estrategias utilizadas. La
idea para probar esto esencialmente fue intentar seguir como modelo las pruebas
realizadas para MDPs en \cite{AlfaroThesis,BaierKatoen}. Esto en el caso del
teorema~\ref{teoEstabilidadEC} vino sin complicaciones, pero en el caso del
teorema~\ref{teoFundEC} devino en problemas, puesto que la finitud de acciones
salientes de un estado probó ser una hipótesis fundamental.

La idea de prueba consiste en proponer un sub-PMDP $(C, D)$ arbitrario que no
cumple la definición~\ref{defEC} de componente final, dividir el análisis por
casos de cómo no se cumple la definición y mostrar que, en cada caso, la
probabilidad de que un camino genere un sub-PMDP como el propuesto es 0.

Remitiéndonos a nuestra prueba, vemos que en el primer ítem el argumento está
explicítamente respaldado en el hecho de que $\pi$ es de memoria finita. De
esta manera, se puede decir que el conjunto de las distintas probabilidades de
quedarse en $C$ en los momentos $i$ es finito, por lo que tiene un máximo $r$,
lo que permite acotar la productoria $\Pi_{i \in I} \ r_i$ por $\lim_{k \to
		\infty} r^k$, que sabemos que converge a 0 puesto que $r < 1$.

Si no nos restringimos a estrategias de memoria finita no podemos garantizar
esto. Sabemos que la productoria va a estar acotada inferiormente por 0 y
superiormente por 1, pero si tengo una cantidad infinita de $r_i$'s distintos,
aún cuando cada uno de ellos es menor a $1$, la productoria bien podría no
converger a $0$, como se puede ver en el siguiente caso.

Consideremos un PMDP en donde tenemos dos estados $t$ y $s$. Desde $t$, existen
acciones $\mu$ de la forma $\mu(s) = \frac{1}{k^2}$, $\mu(t) = 1 -
	\frac{1}{k^2}$, por lo que se puede definir una estrategia de memoria infinita
$\pi$ a partir de la cantidad de $t$ que se encuentran en el camino de la
siguiente manera:

\begin{align*}
	&\pi(\omega t)(s) = \frac{1}{(\#_t(\omega))^2} \\
	&\pi(\omega t)(t) = 1- \frac{1}{(\#_t(\omega))^2}
\end{align*}

donde $\#_t$ se define inductivamente sobre caminos de la siguiente manera:

\begin{align*}
	&\#_t(\emptyset) = 0 \\
	&\#_t(\omega V' t) = \#_t(\omega) + 1 \\
	&\#_t(\omega V' s) = \#_t(\omega)
\end{align*}

siendo $\omega$ un camino, $\emptyset$ el camino vacío y $s \neq t$.

Si suponemos que $s$ no forma parte de $C$, tenemos que la probabilidad de
quedarnos en $C$ sería $\Pi_{i \geq 0} 1-(\frac{1}{i^2})$, que sabemos que
converge a $\frac{1}{2}$.

\begin{figure}[h]
	\centering
	\begin{tikzpicture}[>=Stealth, node distance=3.5cm, font=\scriptsize]

		% Styles for different shapes
		\tikzset{
			state/.style={draw, circle, minimum size=1.2cm, font=\normalsize},
			square state/.style={draw, rectangle, minimum size=1.2cm, font=\normalsize},
			diamond state/.style={draw, diamond, aspect=1.5, minimum size=1.5cm, font=\normalsize}
		}

		% Nodes
		\node[state] (t) at (0,0) {$t$};
		\node[state] (s) at (4,0) {$s$};

		% Transitions
		\draw[->] (t) to[bend left=20] node[above] {$1 - \frac{1}{i^2}$} (s);
		\draw[->] (t) edge[loop left] node[left] {$\frac{1}{i^2}$} (t);
	\end{tikzpicture}
	\caption{Interpretación del PDMDP con la estrategia $\pi$ fijada.}
	\label{fig:pmdpcontraej}
\end{figure}

Ahora bien, es cierto que esta premisa se basa en el hecho de que $s$ no esté
en $C$, que no es algo que podamos afirmar a priori, y que esto que mostramos
solo implica que este método de prueba no es útil para probar el
teorema~\ref{teoFundEC} para cualquier tipo de estrategias; no significa que no
sea válido.

Esta limitación a la que nos tuvimos que atener aquí, sin embargo, no
constituyó una limitación a la prueba del teorema principal sobre regiones
ganadoras que mostramos más adelante, porque pudimos reducir lo que
necesitabamos para su prueba al corolario~\ref{adaptB30}.

% ES PMDP CHE
% Consideremos un juego estocástico politópico donde tenemos un vértice $t$ y un
% vértice $s$. Desde $t$, existen acciones $\mu$ de la forma $\mu(s) =
% 	\frac{1}{k^2}$, $\mu(t) = 1 - \frac{1}{k^2}$, por lo que se puede definir una
% estrategia de memoria infinita $\pi$ a partir de la cantidad de $t$ que se
% encuentran en el camino de la siguiente manera:

% \begin{align*}
% 	&\pi(\omega t)(s) = \frac{1}{(\#_t(\omega))^2} \\
% 	&\pi(\omega t)(t) = 1- \frac{1}{(\#_t(\omega))^2}
% \end{align*}

% donde $\#_t$ se define inductivamente sobre caminos de la siguiente manera:

% \begin{figure}[h]
% 	\centering
% 	\begin{tikzpicture}[>=Stealth, node distance=3.5cm, font=\scriptsize]

% 		% Styles for different shapes
% 		\tikzset{
% 			state/.style={draw, circle, minimum size=1.2cm, font=\normalsize},
% 			square state/.style={draw, rectangle, minimum size=1.2cm, font=\normalsize},
% 			diamond state/.style={draw, diamond, aspect=1.5, minimum size=1.5cm, font=\normalsize}
% 		}

% 		% Nodes
% 		\node[square state] (t) at (0,0) {$t$};
% 		\node[diamond state] (s) at (4,0) {$s$};

% 		% Transitions
% 		\draw[->] (t) to[bend left=20] node[above] {$1 - \frac{1}{i^2}$} (s);
% 		\draw[->] (t) edge[loop left] node[left] {$\frac{1}{i^2}$} (t);
% 	\end{tikzpicture}
% 	\caption{Interpretación del juego estocástico con la estrategia $\pi$ fijada.}
% 	\label{fig:mi-figura}
% \end{figure}

% \begin{align*}
% 	&\#_t(\emptyset) = 0 \\
% 	&\#_t(\omega V' t) = \#_t(\omega) + 1 \\
% 	&\#_t(\omega V' s) = \#_t(\omega)
% \end{align*}

% donde $\omega$ es un camino, $\emptyset$ representa el camino vacío y $s \neq t$.

% Si suponemos que $s$ no forma parte de $C$, tenemos que la probabilidad de quedarnos en $C$ sería
% $\Pi_{i \geq 0} 1-(\frac{1}{i^2})$, que sabemos que converge a $\frac{1}{2}$.

% \hl{No me gustan mucho estos siguientes párrafos, pero no sé muy bien cómo decir lo que quiero, y siento que vos habías dicho algo más interesante sobre esto.}

% sólo se garantiza para estrategias de memoria finita, debido a la dependencia
% de la prueba en la existencia de una cota superior para la probabilidad de
% permanecer en $C$ (el valor $r$), que sólo puede asegurarse cuando el número de
% distribuciones posibles es finito. En el caso de estrategias de memoria
% infinita, la secuencia de probabilidades podría no estar acotada uniformemente
% por debajo de 1, y la productoria correspondiente podría no converger a 0, como
% se ilustra en el ejemplo anterior. Por lo tanto, la generalización del
% resultado a estrategias de memoria infinita requeriría técnicas de prueba
% diferentes o condiciones adicionales sobre la estructura del PMDP o las
% estrategias consideradas.

% La prueba del teorema \ref{teoFundEC} está inspirada en las pruebas realizadas
% para MDPs (véase \cite{BaierKatoen} \cite{AlfaroThesis}). En ellas, la prueba
% consiste en proponer un sub-PMDP $(C, D)$ arbitrario que no cumple la
% definición \ref{defEC}, dividir el análisis por casos de cómo no se cumple la
% definición de componente final y mostrar que en cada caso la probabilidad de
% que un camino que siga la estrategia del enunciado genere un sub-PMDP como el
% propuesto es 0.

% Si nos remitimos a la prueba que mostramos en \ref{teoFundEC}, vemos que el
% método de prueba se ajusta bien a nuestro caso hasta llegar al primer ítem
% donde el argumento está explícitamente respaldado en el hecho de que $\pi$ es
% de memoria finita. De esta manera, se puede decir que el conjunto de las
% distintas probabilidades de quedarse en $C$ en los momentos $i$ tiene un
% máximo, $r$, lo que permite acotar la productoria $\Pi_{i \in I} \ r_i$ por
% $\lim_{k \to \infty} r^k$, que sabemos que converge a 0, puesto que $r < 1$.

% Si no nos restringimos a estrategias de memoria finita no podemos garantizar
% esto. Sabemos que la productoria va a estar acotada inferiormente por 0 y
% superiormente por 1, pero bien podría no converger a 0, sino a algún otro
% número. Por ejemplo, veamos lo siguiente:

% Ahora bien, es importante aclarar que esto no constituye de ninguna manera un
% contraejemplo, además de que faltaría formalidad en su presentación, resulta
% que plantear $s \notin C$ siendo que tengo probabilidad positiva de visitarlo
% cada vez no tiene mucho sentido. (Esto no sé que tanto es así).

% En cualquier caso, podría ser una muestra de que la aplicación de este fórmula
% de prueba no es útil \textit{directamente} para probar el teorema para
% estrategias de memoria infinita. Pero resaltamos el directamente porque la
% productoria constituye una cota a la probabilidad de que $\omega \in
% 	\Paths_s^{(C,D)}$, no es directamente el valor de la probabilidad.



\section{Juegos justos y desrandomización de un PSG}
\label{sec:prueba}

A fin de responder la pregunta cualitativa en el estudio de los juegos
estocásticos politópicos con objetivos de Rabin, presentaremos un tipo nuevo de
juegos deterministas: los juegos justos. Probaremos que el conjunto de vértices
ganadores con probabilidad 1 en un juego estocástico politópico con un objetivo
de Rabin es igual al conjunto de vértices ganadores de un juego justo
construido a partir del PSG.

Para eso primero presentaremos el concepto de juego justo y la construcción del
juego justo a partir del PSG. Luego, explicitaremos un poco más la relación
entre el PSG y el juego justo construido, al cual llamaremos su
desrandomización. Seguido de eso, expondremos la prueba formal de la igualdad
entre los conjuntos. Y, por último, mostraremos un algoritmo para el cálculo de
los estados ganadores y la sintésis de estrategias ganadoras para un PSG con un
objetivo de Rabin. % VER ESTO EN DEPENDENCIA DE LO QUE TERMINEMOS ESCRIBIENDO AL FINAL

\subsection*{Juegos de adversario justo}

Sea $G$ un juego de grafo de dos jugadores y sea $E^{l} \subseteq (V_\diam
	\times V) \cap E$ un conjunto dado de aristas que deben ser tomadas
infinitamente a menudo si el estado del que parten es visitado infinitamente a
menudo. Nombraremos $V^{l} := \mathrm{dom}(E^{l})$ al conjunto de vértices del
jugador $\diam$ que está en el dominio de $E^{l}$. Las aristas en $E^{l}$
representarán suposiciones de equidad sobre el jugador $\diam$: para cada
arista $(v, v') \in E^{l}$, si $v$ es visitado infinitamente a menudo en una
jugada, se espera que la arista $(v, v')$ sea elegida también infinitamente a
menudo por el jugador $\diam$. Es decir, si un vértice $v$ es visitado
infinitas veces, se espera también que toda arista en $E^l$ saliente de $v$ sea
tomada una cantidad infinita de veces.

Denotamos por $G^{l} = \langle G, E^{l} \rangle$ a un juego de adversario
justo, y extendemos nociones como jugadas, estrategias, condiciones ganadoras,
regiones ganadoras, etc. de juegos deterministas de manera natural. Una jugada
$\rho$ sobre $G^{l}$ se dice fuertemente justa si satisface la siguiente
fórmula en lógica temporal lineal (LTL):
\[
	\alpha := \bigwedge_{(v, v') \in E^{l}} \left( \siempevent v \rightarrow \siempevent (v \wedge \bigcirc v') \right).
\]
Dado $G^{l}$ y una condición de victoria $\varphi$, el jugador $\cuad$ gana el
juego de adversario justo sobre $G^{l}$ con respecto a la condición de victoria
$\varphi$ desde un vértice $v_0 \in V$ si gana el juego sobre $G^{l}$ para la
condición de victoria $\alpha \rightarrow \varphi$ desde $v_0$.

Hay dos observaciones interesantes para hacer sobre los juegos de adversario
justo:

Primero, las aristas en $E^l$ permiten descartar ciertas estrategias del
jugador $\diam$, facilitando que el jugador $\cuad$ gane en determinadas
situaciones. Por ejemplo, consideremos un grafo de juego
(figura~\ref{fig:juego-justo}, parte superior) con dos vértices $p$ y $q$. El
vértice $p$ pertenece al jugador $\diam$ y el vértice $q$ al jugador $\cuad$.
La arista $(p, q)$ es una arista en $E^l$ (representada con línea discontinua).
Supongamos que la especificación para el jugador $\cuad$ es $\varphi =
	\siempevent q$. Si la arista $(p, q)$ no estuviera en $E^l$, el jugador $\cuad$
no ganaría desde $p$, porque el jugador $\diam$ podría mantener el juego
atrapado en $p$ eligiéndose a sí mismo como sucesor en cada turno. En
contraste, el jugador $\cuad$ gana desde $p$ en el juego adversarial justo,
porque la suposición de equidad sobre la arista $(p, q)$ fuerza al jugador
$\diam$ a elegir infinitamente a menudo la transición hacia $q$.

Segundo, las suposiciones de equidad modeladas por aristas en $E^l$ restringen
las elecciones de estrategias del jugador $\diam$ menos que lo que
restringirían la suposición de que el jugador $\diam$ elige probabilísticamente
entre estas aristas. Consideremos, por ejemplo, un juego de adversario justo
con un único vértice del jugador $\diam$, $p$ con dos aristas en $E^l$
salientes hacia los estados $q$ y $q'$, como se muestra en la Figura 1 (parte
inferior). Si el jugador $\diam$ elige aleatoriamente entre las aristas $(p,
	q)$ y $(p, q')$, toda secuencia finita de visitas a los estados $q$ y $q'$
ocurrirá infinitamente a menudo con probabilidad uno. Esto no es cierto en el
juego de adversario justo. Aquí el jugador $\diam$ puede elegir una secuencia
particular de visitas a $q$ y $q'$ (por ejemplo, simplemente $qq'qq'qq'\dots$),
siempre que ambos sean visitados infinitamente a menudo.

\begin{figure}[ht]
	\centering
	\begin{tikzpicture}[>=Stealth, node distance=3.5cm, font=\scriptsize]
		% Definición de estilos para los nodos
		\tikzset{
			state/.style={draw, circle, minimum size=1.2cm, font=\normalsize},
			square state/.style={draw, rectangle, minimum size=1.2cm, font=\normalsize},
			diamond state/.style={draw, diamond, aspect=1.5, minimum size=1.5cm,font=\normalsize}
		}

		% --- Figura superior ---.
		\node[diamond state] (p1) at (0,0) {$p$};
		\node[square state] (q1) [right of=p1] {$q$};

		% Transiciones:
		\draw[->] (p1) edge[loop left] (p1);
		\draw[->, dashed] (p1) to[bend left=20] (q1);
		\draw[->] (q1) edge[bend left=20] (p1);

		% --- Figura inferior ---
		\node[diamond state] (p2) at (0,-3.5) {$p$};
		\node[square state] (q2) [above right=0.5cm and 1cm of p2] {$q$};
		\node[square state] (q2p) [below right=0.5cm and 1cm of p2]  {$q'$};

		% Transiciones:
		\draw[->, dashed] (p2) to[bend left=20] (q2);
		\draw[->, dashed] (p2) edge[bend left=20] (q2p);
		\draw[->] (q2) edge[bend left=20] (p2);
		\draw[->] (q2p) edge[bend left=20] (p2);

	\end{tikzpicture}
	\caption{Dos juegos de adversario justo.}
	\label{fig:juego-justo}
\end{figure}

\subsection*{Desrandomización de PSGs}

Dada la interpretación de un juego estocástico politópico $\Gk = (\St,
	(\St_\cuad, \St_\diam), \Act, \theta)$, se define la \textit{desrandomización
	de} $\Gk$, $\DGk := (( V, V_\cuad, V_\diam, E ), E^l )$, como el juego de grafo
de 2 jugadores justo con:

\begin{alignat*}{2}
	&\tilde V &&= \newV \\
	&V &&= \St \cup \tilde V \\
	&V_\cuad &&= \St_\cuad \\
	&V_\diam &&= \St_\diam \cup \tilde V \\
	&E &&= \{(s, v_{V}) : V \in V_s\} \\
	&E^l &&= \{(v_{V}, s') : s' \in V\}
\end{alignat*}

y para el cual se cumple la siguiente condición de equidad: $$ \varphi^l :=
	\bigwedge_{(v_{V'},s') \in E^l} (\siempevent v_{V'} \rightarrow \siempevent
	(v_{V'} \wedge \bigcirc s' )) $$

Esta misma condición de equidad se puede expresar como

\begin{center}
	$ \varphi^l : = \bigwedge_{v_V \in \tilde V} \varphi^V $, donde

	$ \varphi^V := \bigwedge_{s' \in V} (\siempevent v_{V'} \rightarrow \siempevent (v_{V'} \wedge \bigcirc s' ))$
\end{center}

\textbf{Roborta vs Rigoborto desrandomizado}

Con lo definido en la subsección anterior, podemos ver cómo quedaría el PSG que
presentamos como ejemplo en \ref{sec:ejpsg}.

Para ello, primero deberíamos pensar cuáles son los conjuntos $V_s$ para cada
$s \in \St$. Notaremos a los estados por las coordenadas en las que se
encuentra Roborta arriba y Rigoborto abajo, al igual que hicimos para
graficarlos, y con una $c$ para indicar que pertenecen al jugador $\cuad$ y una
$d$ para indicar que pertenecen al jugador $\diam$. Entonces estos serían
algunos de los conjuntos soportes de diferentes estados que nos son de interés:
% Si nombramos como $s^1$ al estado perteneciente al jugador $\cuad$
% para las posiciones $(0,0)$ de Roborta y $(1,0)$ de Rigoborto, con $s^2$, $s^3$
% y $s^4$ a los estados del jugador $\diam$ para las posición $(1,0)$ de
% Rigoborto y las posiciones $(0,1)$, $(1,1)$ y $(1,0)$ de Roborta,
% respectivamente, y con $s^5$, $s^6$, $s^7$, y tenemos los siguientes conjuntos
% soportes para cada uno:
\begin{itemize}
	\item $V_{\substack{c(0,0) \\ \ (1,0)}} = \left\{\left\{\substack{d(0,1) \\ \ (1,0)}, \substack{d(1,1) \\ \ (1,0)}, \substack{d(1,0) \\ \ (1,0)}\right\}\right\}$
	\item $V_{\substack{d(0,1) \\ \ (1,0)}} = \left\{\left\{\substack{c(0,1) \\ \ (0,0)}\right\}, \left\{\substack{c(0,1) \\ \ (1,1)}\right\}\right\}$
	\item $V_{\substack{d(1,1) \\ \ (1,0)}} = \left\{\left\{\substack{c(1,1) \\ \ (0,0)}\right\}, \left\{\substack{c(1,1) \\ \ (1,1)}\right\}\right\}$
	\item $V_{\substack{d(1,0) \\ \ (1,0)}} = \left\{\left\{\substack{c(1,0) \\ \ (0,0)}\right\}, \left\{\substack{c(1,0) \\ \ (1,1)}\right\}\right\}$
\end{itemize}

Entonces, si quisiesemos desrandomizar el fragmento de juego estocástico que
vimos en la figura~\ref{fig:psg} quedaría como muestra la
figura~\ref{fig:desrand}.

\begin{figure}[ht]
	\centering
	\begin{tikzpicture}[->]
		\tikzset{
			state/.style={draw, circle, minimum size=1.2cm, font=\normalsize},
			square state/.style={draw, rectangle, minimum size=1.4cm, font=\normalsize, align=center},
			diamond state/.style={draw, diamond, aspect=1.5, minimum size=1.5cm, font=\normalsize, align=center},
			every node/.style={inner sep=4pt}
		}

		% Nodo raíz a la izquierda
		\node[square state] (left) at (0, 0) {$(0,0)$\\$(1,0)$};

		% Nodo Vleft
		\node[diamond state] (Vleft) at (3,0) {$V^1_{\substack{c(0,0) \\ \ (1,0)}}$};

		% Nivel intermedio (rombos) centrados horizontalmente, espaciados verticalmente
		\node[diamond state] (d1) at (6, 4) {$(0,1)$\\$(1,0)$};
		\node[diamond state] (d2) at (6, 0) {$(1,1)$\\$(1,0)$};
		\node[diamond state] (d3) at (6, -4) {$(1,0)$\\$(1,0)$};

		% Nodos Vd
		\node[diamond state] (Vs1) at (9, 5) {$V^1_{\substack{d(0,1) \\ \ (1,0)}}$};
		\node[diamond state] (Vs2) at (9, 3) {$V^2_{\substack{d(0,1) \\ \ (1,0)}}$};

		\node[diamond state] (Vs3) at (9, 1.0) {$V^1_{\substack{d(1,1) \\ \ (1,0)}}$};
		\node[diamond state] (Vs4) at (9, -1.0) {$V^2_{\substack{d(1,1) \\ \ (1,0)}}$};

		\node[diamond state] (Vs5) at (9, -3) {$V^1_{\substack{d(1,0) \\ \ (1,0)}}$};
		\node[diamond state] (Vs6) at (9, -5) {$V^2_{\substack{d(1,0) \\ \ (1,0)}}$};

		% Nodos hojas (cuadrados), suficientemente separados
		\node[square state] (s1) at (12, 5) {$(0,1)$\\$(0,0)$};
		\node[square state] (s2) at (12, 3) {$(0,1)$\\$(1,1)$};

		\node[square state] (s3) at (12, 1.0) {$(1,1)$\\$(0,0)$};
		\node[square state] (s4) at (12, -1.0) {$(1,1)$\\$(1,1)$};

		\node[square state] (s5) at (12, -3) {$(1,0)$\\$(0,0)$};
		\node[square state] (s6) at (12, -5) {$(1,0)$\\$(1,1)$};

		% Arista de left a Vleft
		\draw[-> ] (left) to (Vleft);

		% Aristas desde left a cada rombo
		\draw[-> ] (Vleft) to (d1);
		\draw[-> ]  (Vleft) to (d2);
		\draw[-> ] (Vleft) to (d3);

		% Aristas desde cada rombo a dos Vs
		\draw[-> ] (d1) to (Vs1);
		\draw[-> ] (d1) to (Vs2);

		\draw[-> ] (d2) to (Vs3);
		\draw[-> ] (d2) to (Vs4);

		\draw[-> ] (d3) to (Vs5);
		\draw[-> ] (d3) to (Vs6);

		% Aristas desde cada Vs a cuads
		\draw[-> ] (Vs1) to (s1);
		\draw[-> ] (Vs2) to (s2);

		\draw[-> ] (Vs3) to (s3);
		\draw[-> ] (Vs4) to (s4);

		\draw[-> ] (Vs5) to (s5);
		\draw[-> ] (Vs6) to (s6);

		% Puntos suspensivos para indicar que continúa
		\node at (13,5) {$\cdots$};
		\node at (13,3) {$\cdots$};

		\node at (13,1) {$\cdots$};
		\node at (13,-1) {$\cdots$};

		\node at (13, -3) {$\cdots$};
		\node at (13, -5) {$\cdots$};

	\end{tikzpicture}
	\caption{Desrandomización del PSG de Roborta y Rigoborto.}
	\label{fig:desrand}
\end{figure}

\subsection*{Relación entre un PSG y su desrandomización}

Nos será útil pensar en transformaciones entre los distintos juegos para la
prueba que tenemos más adelante, así que veremos algunas definiciones. Antes,
fijemos la interpretación de un juego estocástico politópico $\Gk$ y su
desrandomización $\DGk$.

\textbf{Condición de equidad en el juego estocástico politópico}

Podemos expresar fácilmente la condición de equidad en el juego estocástico
politópico de la siguiente forma:

\begin{center}
	$
		\hat \varphi^l := \bigwedge_{\alpha \in \Act} \hat \varphi^{\supp(\alpha)}
	$, con

	$
		\hat \varphi^{\supp(\alpha)} := \bigwedge_{s' \in \supp(\alpha)} (\siempevent \alpha \rightarrow \siempevent (\alpha \wedge \bigcirc s' ))
	$
\end{center}

\kathy{Ver si agg lo de condiciones de victoria y capaz agg lo de Rabin}

\textbf{Transformación de caminos}

Dado un camino $\omega = (s_0, \alpha_0, s_1, \alpha_1, \dots)$ en $\Gk$,
podemos obtener un único camino en $\DGk$ al que llamaremos $\derand(\omega)$ y
tendrá la siguiente forma: $$\derand(\omega) = (s_0, v_{\supp(\alpha_0)}, s_1,
	v_{\supp(\alpha_1), \dots})$$ Por otro lado, si tenemos un camino $\rho = (s_0,
	v_{V_0}, s_1, v_{V_1}, \dots)$ en $\DGk$, existen varios caminos en $\Gk$ que
se corresponderían con él. Llamaremos a este conjunto de caminos $\rand(\rho)$.
Es decir, $$\rand(\rho) = \{(s_0, \alpha_0, s_1, \alpha_1, \dots) \in
	\Omega_{\Gk,s} \mid \forall i \geq 0, \ \supp(\alpha_i) = V_i\}$$

Estas definiciones también se extienden a prefijos finitos de caminos de manera
natural.

\textbf{Randomización de estrategias}

Supongamos que tenemos una estrategia $\sigma_i$ en $\DGk$ y queremos construir
una estrategia $\pi_i$ en $\Gk$ que tenga un comportamiento similar. $\sigma_i$
está definida para todos los prefijos finitos de caminos $\rho$ en $\DGk$ y
nuestra idea es definir $\pi_i$ para todos los prefijos finitos de camino
$\omega$ en $\Gk$. La idea, entonces, será definir $\pi_i$ de igual manera para
todos los elementos del conjunto $\rand(\rho)$.

Solo nos va a interesar ver cómo $\sigma_i$ se comporta en los prefijos finitos
de caminos $\rho$ que terminen en un estado $s \in \St$ \footnote{Desde los
	prefijos finitos de caminos donde el último elemento es un vértice $v_V$, las
	decisiones son tomadas por el jugador $\diam$ en $\DGk$, pero solo reflejan lo
	que sería la decisión probabilística en $\Gk$.}. Para cada uno de estos $\rho$,
tenemos que $\sigma_i(\rho) = v_{\widehat{V}}$ para algún $v_V$. Lo que haremos
para la construcción de $\pi_i$ es para cada $\omega \in \derand(\rho)$ definir
$\pi_i(\omega)(\hat\alpha) = 1$ para algún $\hat\alpha$ particular tal que
$\supp(\hat\alpha) = \widehat{V}$.

Llamaremos a esta nueva estrategia $\pi_i$ obtenida a partir de $\sigma_i$,
$\rand(\sigma_i)$.

\begin{center}
	$\sigma_i(s_0, v_{V_0}, \dots, s_k) = v_{V_k} \implies \rand(\sigma_i)(s_0, \alpha_0, \dots, s_k)(\alpha_k) = 1$ donde \\ $\forall i \ \supp(\alpha_i) = V_i$.
	\captionof{figure}{Una forma de ver la randomización de estrategias en $\DGk$}
\end{center}
% \begin{figure}[ht]
% 	\centering
% 	\begin{equation*}
% 		\sigma_i(s_0, v_{V_0}, \dots, s_k) = v_{V_k} \implies
% 		\rand(\sigma_i)(s_0, \alpha_0, \dots, s_k)(\alpha_k) = 1
% 	\end{equation*}
% 	\caption{Una forma de ver la randomización de estrategias}
% \end{figure}

\textbf{Desrandomización de estrategias}

Supongamos ahora que tenemos una estrategia $\pi_i$ en $\Gk$ y queremos
construir una estrategia $\sigma_i$ en $\DGk$ que tenga un comportamiento
similar.

Para cada prefijo finito de camino $\rho$ debemos definir qué vértice será el
que elija $\sigma_i(\rho)$. La idea será que elegiremos un vértice $v_V$ que se
corresponde al soporte de una acción a la que $\pi_i(\omega)$ (siendo $\omega$
tal que $\derand(\omega) = \rho$) le asigne una probabilidad positiva.

Como existen varios prefijos de camino que podrían cumplir con la condición de
$\omega$, la definición de $\derand(\pi_i)$ dependerá de la elección de un
prefijo de camino $\omega$ para cada prefijo de camino $\rho$ tal que
$\derand(\omega) = \rho$.

Además, como cada acción le puede dar probabilidad positiva a varias acciones
con distintos soportes, la definición de $\derand(\pi_i)$ también dependerá de
la elección de una acción a la que $\pi_i(\omega)$ le asigne una probabilidad
positiva (por cada $\omega$ seleccionado en el paso anterior).

\begin{boxgris}[¿Cómo se podrían tomar esas elecciones?]{}
	Suponiendo que tenemos una estrategia $\pi_i$ en $\Gk$, es probable que a su vez tengamos caminos $\omega'_1, \omega'_2, \dots$ en $\Gk$ específicos en que estemos interesados que $\sigma_i$ replique. En ese caso, una manera de tomar las decisiones antes planteadas es usando los prefijos finitos de los distintos $\omega'_k$ para las decisiones de $\sigma_i$. Esto sería: si tenemos un prefijo de camino $\hat \omega = (s_1, \alpha_1, \dots, s_m, \alpha_m)$ (que respeta $\pi_i$), entonces haremos que para $\hat \rho = (s_1, v_{\supp(\alpha_1)}, \dots, s_m)$, $\sigma_i(\hat \rho) = v_{\supp(\alpha_m)}$.
\end{boxgris}

Con estas decisiones tomadas, podemos definir como se comportará $\sigma_i$
para cada prefijo de camino que termina en un estado $s \in \St$.

Ahora bien, para el caso de $\sigma_\diam$ también hay que definir cómo se
comporta la estrategia en los caminos que terminan en los vértices de la forma
$v_V$. Es decir, debemos elegir qué estado $s_{k+1}$ será el que cumpla
$\sigma_\diam(s_0, v_{V_0}, \dots, s_k, v_{V_k}) = s_{k+1}$ para cada $k$. En
este caso, lo que nos debemos asegurar es que se cumpla la condición de
equidad.

% Se podría pensar como tener esto aparte o enmarcado en una box. Se podría agregar otra box con lo que sería medio la precontinuacion del punto 2
\begin{boxgris}[¿Cómo asegurar la condición de equidad?]{}
	Presentamos dos maneras fáciles con las cuales se podría asegurar esto, pero
	podrían existir infinidades de ellas:
	\begin{enumerate}
		\item como cada $V_k$ es finito, podemos numerar cada $s^{V_k} \in V_k$ de manera
		      $s_0^{V_k}, s_1^{V_k}, \dots$. Esto nos permite seleccionar el próximo estado
		      en base a la cantidad de veces que se visitó $v_{V_k}$. Si $n$ fueron las veces
		      que se visitó $v_{V_k}$ en el prefijo finito de camino $\rho '$, entonces
		      podemos definir $\sigma_\diam(\rho ' v_{V_k}) = s_n^{V_k}$.
		\item si tenemos un camino $\hat \omega$ válido que respeta la estrategia
		      $\pi_\diam$, y a partir de los prefijos de este $\hat \omega$ es que estamos
		      definiendo las elecciones de prefijos y acciones de $\sigma_\diam$, podemos
		      también usar este $\hat \omega$ para las decisiones desde los estados
		      $v_{V_k}$, eligiendo como próximo vértice $s_{k+1}$ al que se eligió
		      probabilísticamente en $\hat \omega$.
	\end{enumerate}
\end{boxgris}

Con estas decisiones ya tomadas resulta la construcción de la $\sigma_i$ que
queríamos, a la cual llamaremos $\derand(\pi_i)$.

\textbf{Respetar una estrategia}

Al hablar de caminos específicos es natural pensar que estos fueron producto de
una partida en la que los jugadores siguieron determinadas estrategias. Para la
prueba que plantearemos a continuación querremos formalizar esta relación entre
caminos y las estrategias que se siguieron en ellos y lo haremos a través de la
idea de que un camino ``respeta" determinadas estrategias.

Sea $\omega = (s_0, \alpha_0, s_1, \alpha_1, \dots)$ un camino en un juego
estocástico poliópico y sean $\picuad$ y $\pidiam$ dos estrategias en el mismo
juego. Decimos que $\omega$ respeta las estrategias $\picuad$ y $\pidiam$ si
$\forall i \geq 0$ vale
\begin{align*}
	&\bigl( s_i \in \St_\cuad \wedge \picuad(s_0, \alpha_0, \dots, s_i)(\alpha_i) > 0 \wedge s_{i+1} \in \supp(\alpha_i) \bigr) \bigvee \\
	&\left(s_i \in \St_\diam \wedge \pidiam(s_0, \alpha_0, \dots, s_i)(\alpha_i) > 0 \wedge
	s_{i+1} \in \supp(\alpha_i) \right)
\end{align*}
% \begin{align*}
% 	&\bigl( s_i \in \St_\cuad \wedge \picuad(\hat
% 	\omega_i)(\alpha_i) > 0 \wedge s_{i+1} \in \supp(\alpha_i) \bigr) \bigvee \\
% 	&\left(s_i \in \St_\diam \wedge \pidiam(\hat \omega_i)(\alpha_i) > 0 \wedge
% 	s_{i+1} \in \supp(\alpha_i) \right)
% \end{align*}

Sea $\rho = (s_0, v_{V_0}, s_1, v_{V_1}, \dots)$ un camino en la
desrandomización de un juego estocástico politópico y sean $\sigma_\cuad$ y
$\sigma_\diam$ dos estrategias en el mismo juego. Decimos que $\rho$ respeta
las estrategias $\sigma_\cuad$ y $\sigma_\diam$ si $\forall i \geq 0$ vale

\begin{align*}
	\sigma_\diam&(s_0, v_{V_0}, \dots, s_i, v_{V_i})=s_{i+1} \bigwedge \\
	&\bigl( \left( s_i \in \St_\cuad \wedge \sigma_\cuad(s_0, v_{V_0}, \dots, s_i)=v_{V_i} \right) \bigvee \\
	&\left( s_i \in \St_\diam \wedge \sigma_\diam(s_0, v_{V_0}, \dots, s_i)=v_{V_i} \right) \bigr)
\end{align*}
%$ \sigma_\diam(\hat \rho_{v_{V_i}})=s_{i+1} \bigwedge \left( \left( s_i \in \St_\cuad \wedge \sigma_\cuad(\hat \rho_{s_i})=v_{V_i} \right) \bigvee \left( s_i \in \St_\diam \wedge \sigma_\diam(\hat \rho_{s_i})=v_{V_i} \right) \right)$

\section{Prueba de igualdad sobre los conjuntos ganadores}

\begin{theorem}
	\label{teocuali}
	Sea $\Gk = (\St, (\St_\cuad, \St_\diam), \Act, \theta)$ la interpretación de un juego estocástico politópico, $R = \{(E_1, F_1), \dots, (E_d, F_d)\}$ una condición de Rabin sobre $\St$, con su especificación LTL $\varphi$,

	$$
		\varphi := \bigvee_{j \in [1, k]} \left( \eventsiemp \overline{E_j} \wedge \eventsiemp F_j \right)
	$$

	y sea $\DGk$ su desrandomización.

	Sea $\W \subseteq \St$ el conjunto de todos los estados desde los cuales el
	jugador $\cuad$ gana en $\DGk$ y sea $\W^{as}$ el conjunto de vértices desde
	los cuales el jugador $\cuad$ gana con probabilidad 1 en $\Gk$. Entonces, $\W =
		\W^{as}$.

	Es más, a partir de una estrategia ganadora en $\DGk$ se puede construir
	fácilmente una estrategia ganadora en $\Gk$, y viceversa.
\end{theorem}

\begin{proof}
	Probaremos la doble contención:

	\textbf{Primera contención: } $\W \subseteq \W^{as}$

	Sea $s \in \W$. Entonces, sabemos que existe al menos una estrategia del
	jugador $\cuad$ ganadora desde $s$ en $\DGk$. Llamemos a esta
	$\sigma_\cuad^{*}$.

	Queremos ver que $s \in \W^{as}$, lo que requiere ver que existe una estrategia
	$\picuadGk$ tal que:
	\begin{equation}
		\label{pisirve}
		\inf_{\pidiam \in \Pidiam} \Prob_{\Gk,s}^{\picuadGk, \pidiam}(\varphi) = 1
	\end{equation}

	Proponemos $\picuadGk = \rand(\sigma_\cuad^*)$ y veremos que vale \ref{pisirve}
	por reducción al absurdo.

	Supongamos que no vale \ref{pisirve}, es decir,
	\begin{equation}
		\inf_{\pidiam \in \Pidiam} \Prob_{\Gk, s}^{\picuadGk, \pidiam} (\varphi) < 1.
	\end{equation}

	Esto significa que existe una estrategia $\pidiam^*$ que hace que $\Prob_{\Gk,
			s}^{\picuadGk, \pidiam} (\varphi) < 1$. A su vez, esto significa que existe un
	camino $\omega^*$ que empieza desde $s$, respeta las estrategias $\picuadGk$ y
	$\pidiam^*$, pero no cumple $\varphi$.

	Pensemos, entonces, en $\sigma_\diam^*$ una desrandomización válida de
	$\pidiam^*$ que sigue las elecciones de acciones que toman los prefijos de
	$\omega^*$.

	Luego, $\rho^* = \derand(\omega^*)$ respeta las estrategias $\sigma_\cuad^*$ y
	$\sigma_\diam^*$. Como $\rho^* \cap \St = \omega^* \cap \St$ y $\omega^*$ no
	cumple $\varphi$, $\rho^*$ no cumple $\varphi$. Es decir, existe un camino que
	empieza desde $s$, respeta $\sigma_\cuad^*$, pero no es ganador para $\varphi$.
	Esto contradice que la estrategia $\sigma_\cuad^*$ sea ganadora desde $s$.

	Llegamos a esta contradicción por suponer que no vale \ref{pisirve}. Entonces
	la ecuación sí vale y, consecuentemente, tenemos que $s \in \W^{as}$, como
	queríamos probar.
	% -------------------------------------------------------------------------------------------------------------------------------------------------

	\textbf{Segunda contención: } $\W^{as} \subseteq \W$

	Sea $s \in \W^{as}$, veamos que $s \in \W$.

	Como $s \in \W^{as}$, entonces existe $\picuad^*$ en $\Gk$ tal que
	$\inf_{\pidiam \in \Pidiam}\Prob_{\Gk,s}^{\picuad^*, \pidiam}(\varphi) = 1$.

	Lo que queremos ver para probar que $s \in \W$ es que existe una estrategia
	$\sigma_\cuad^*$ tal que $\cuad$ gana con ella desde $s$ en $\DGk$.

	Definimos $\sigma_\cuad^* = \derand(\picuad^*)$

	Entonces ahora queremos ver que $\sigma_\cuad^*$ es ganadora en $\DGk$ desde
	$s$.

	Una manera de ver esto es probar que para un camino cualquiera $\rho$ generado
	por $\sigma_\cuad^*$ y una estrategia válida $\sigma_\diam$ arbitraria en
	$\DGk$, $\Inf(\rho)_C = \Inf(\rho) \cap \St$ cumple con la especificación
	$\varphi$. Es decir, que vale la siguiente fórmula:

	\begin{equation}
		\label{varphi}
		\varphi' := \bigvee_{j = 1}^{k} \left((\Inf(\rho)_C \cap R_j = \emptyset) \wedge (\Inf(\rho)_C \cap G_j \neq \emptyset)  \right)
	\end{equation}

	Como $\picuad^*$ es una estrategia ganadora frente a cualquier estrategia
	$\pidiam$, siempre se cumple la condición de Rabin con probabilidad 1 en el
	PMDP $\M'$ que se forma al fijar la estrategia $\picuad^*$ en $\Gk$. Por
	teorema~\ref{adaptB30}, entonces vale que para cada componente final $(C,D)$
	alcanzable desde $s$ en el PMDP $\M^*$ vale que existe algún $j \in [1,d]$ tal
	que $C \cap E_j = \emptyset$ y $C \cap F_j \neq \emptyset$.

	Si denotamos como $\Inf(\rho)_D$ a la función que le asocia a cada $s_i$ en
	$\Inf(\rho)_C$ el conjunto formado solo por el $V_i$ correspondiente al vértice
	$v_{V_i}$ en $\rho$, esto que presentamos anteriormente quiere decir que si
	podemos probar que $(\Inf(\rho)_C, \Inf(\rho)_D)$ es una componente final
	alcanzable desde $s$ en el PMDP $\M^*$, entonces vale la ecuación~\ref{varphi}.

	% probar que es subPMDP no hace falta pq es solo una cuestion de tipado

	Veamos entonces primero que $(\Inf(\rho)_C, \Inf(\rho)_D)$ cumple las dos
	condiciones para ser componente final en $\M^*$:

	% Si podemos probar que $\Inf(\rho)'$ es un sub-PMDP alcanzable en el PMDP que se
	% forma al fijar la estrategia de memoria finita $\picuad^*$ en $\Gk$, por el
	% teorema \ref{adaptB30} podemos ver que vale la ecuación \ref{varphi}.

	% Primero, podemos ver que $\inft(\rho)$ \hl{(entendiendo los vértices de la
	% 	forma $v_V$ como el conjunto resultado $V$)} es una componente final en el
	% polytopal markov decision process que se forma al fijar la estrategia
	% $\picuad^*$ en $\Gk$, llamémoslo $(C,D)$, viendo que cumple las dos
	% propiedades:

	\begin{itemize}
		\item Para cada $V^i$ que aparece como subíndice de vértices \textit{especiales} en
		      $\Inf(\rho)$, vale que $V^i \subseteq C$. En el caso de que existiese algún
		      $s_x \in V^i$ tal que $s_x \notin C$, se estaría contradiciendo que $\pidiam$
		      sea una estrategia válida en $\DGk$, puesto que visitaría infinitas veces
		      $v_{V^i}$ y solo finitas veces $s_x$, uno de sus sucesores.
		\item El grafo dirigido inducido por $\Inf(\rho)$ es fuertemente conexo. Si fuese de
		      otra manera habría dos vértices $u, v \in \Inf(\rho)$ tales que $v$ no sería
		      alcanzable desde $u$, contradiciendo así que $u$ y $v$ son visitados infinitas
		      veces por $\sigma$.
	\end{itemize}

	Luego, podemos ver que $\Inf(\rho)_C$ es alcanzable en $\Gk$ viendo que existe
	una estrategia $\pidiam^*$ en $\Gk$ que lo posibilita.

	Definamos $\pidiam^* = \rand(\sigma_\diam)$.

	Esta estrategia permite llegar con probabilidad positiva a $\Inf(\rho)_C$,
	puesto que tanto $\pidiam^*$ como $\picuad^*$ les dan probabilidad positiva a
	los mismos vértices en $\St$ que $\sigma_\diam$ asegura visitar infinitamente.

	Con lo cual hemos probado que vale l ecuación~\ref{varphi} y, por lo tanto, que
	$\sigma_\cuad*$ es ganadora en $\DGk$ desde $s$, con lo que hemos probado que
	$s \in \W$.

\end{proof}

\section{Implicancias algorítmicas de la prueba}

\subsection*{Algoritmo para el cálculo de estados ganadores en un juego justo con objetivo de Rabin}

En \cite{Banerjee} se presenta un algoritmo para calcular el conjunto de
vértices ganadores en un juego de adversario justo $G$ con un objetivo de Rabin
$R$. Este algoritmo es de ejecución simbólica, punto fijo

En \cite{Banerjee} se presenta un algoritmo para calcular el conjunto de
vértices ganadores en un juego de adversario justo \( G \) con un objetivo de
Rabin \( R \). Este algoritmo se basa en la caracterización del conjunto
ganador mediante una expresión de punto fijo y es de ejecución simbólica, es
decir, se plantea sobre conjuntos, y operará utilizando transformadores de
estados adecuados.

El algoritmo se expresa en $\mu$-cálculo, una lógica que permite representar
propiedades sobre sistemas de transición finitos mediante operadores de punto
fijo. Formalmente, las fórmulas del $\mu$-cálculo se definen inductivamente
como sigue:
\[
	\varphi ::= p \mid X \mid \varphi \cup \varphi \mid \varphi \cap \varphi \mid \text{pre}(\varphi) \mid \mu X.\varphi \mid \nu X.\varphi
\]
donde \( p \) denota un conjunto de vértices (una proposición atómica), \( X \)
es una variable, \(\text{pre}\) representa un operador monotónico sobre
conjuntos que variará en el cojunto de los transformadores que presentaremos a
continuación ($\{\Pre_\cuad^\exists, \Pre_\diam^\forall, \Cpre, \Lpre^\exists,
	\Apre\}$), y \( \mu X.\varphi \) (resp. \( \nu X.\varphi \)) representa el
menor (resp. mayor) punto fijo de la función definida por \( \varphi \)
respecto de la variable \( X \).

Sea \( G = (V, E, V_0, V_1) \) un grafo de juego, donde \( V \) es el conjunto
de vértices, \( E \subseteq V \times V \) es el conjunto de transiciones, y \(
V_0 \), \( V_1 \) son los vértices controlados por el jugador \( \square \) y
por el adversario \( \diamondsuit \), respectivamente. Se asume además un
subconjunto \( V^\ell \subseteq V_1 \) de vértices \emph{live}, sobre los
cuales se impone una hipótesis de equidad.
% ESTO REESCRIBIR PERO DARLE NOMBRE A LOS VERTICES LIVE -> hacer algo de newV y con eso hablar de los live vertices

Dado un conjunto de vértices \( S \subseteq V \), se definen los siguientes
operadores: % relacionarlo con lo de que quede de set transformers en el parrafo de mu calculo

\begin{itemize}
	\item \textbf{Predecesor existencial del jugador \( \cuad \):}
	      \[
		      \Pre_\cuad^\exists(S) := \{v \in V_\cuad \mid \exists v' \in S \text{ tal que } (v, v') \in E\}
	      \]

	\item \textbf{Predecesor universal del jugador \( \diam \):}
	      \[
		      \Pre_\diam^\forall(S) := \{v \in V_\diam \mid \forall v' \in V,\ (v, v') \in E \Rightarrow v' \in S\}
	      \]

	\item \textbf{Predecesor controlable:}
	      \[
		      \Cpre(S) := \Pre_\cuad^\exists(S) \cup \Pre_\diam^\forall(S)
	      \]

	\item \textbf{Predecesor existencial sobre vértices live:}
	      \[
		      \Lpre^\exists(S) := \{v \in V^\ell \mid \exists v' \in S \text{ tal que } (v, v') \in E\}
	      \]

	\item \textbf{Predecesor casi seguro:}
	      \[
		      \Apre(S, T) := \Cpre(T) \cup \left( \Lpre^\exists(T) \cap \Pre_1^\forall(S) \right)
	      \]
\end{itemize}

Este último operador identifica los vértices desde los cuales el jugador \(
\square \) puede garantizar la llegada a \( T \) sin abandonar \( S \), ya sea
directamente mediante \( \Cpre \), o confiando en la equidad del adversario
para transitar, desde un vértice live, hacia \( T \), manteniéndose mientras
tanto dentro de \( S \).

Dado un objetivo de Rabin especificado por un conjunto de pares \( R = \{(G_i,
R_i)\}_{i \in [1;k]} \), el conjunto de vértices ganadores se obtiene como la
interpretación de la siguiente fórmula del $\mu$-cálculo: % no es esta la manera en la representamos los objetivos de rabin asi que va a haber que cambiar eso

\[
	Z^* = \nu Y_0.\, \mu X_0.\, \bigcup_{p_1 \in P}\, \nu Y_1.\, \mu X_1.\, \bigcup_{p_2 \in P \setminus \{p_1\}}\, \cdots\, \bigcup{p_k} \nu Y_d.\, \mu X_d.\, \left[ \bigcup_{j=0}^d\, C_{p_j} \right]
\]

donde \( P = \{1, \dots, k\} \), y cada término \( C_{p_j} \) se define como:

\[
	C_{p_j} := \left( \bigcap_{i=0}^j R_{p_i} \right) \cap \left[ (G_{p_j} \cap \Cpre(Y_{p_j})) \cup \Apre(Y_{p_j}, X_{p_j}) \right]
\]

con la convención de que \( R_{p_0} = G_{p_0} = \emptyset \). Esta fórmula
alterna puntos fijos máximos y mínimos para reflejar la semántica de la
condición de Rabin, y extiende el algoritmo de Piterman para juegos
deterministas mediante la incorporación del operador \( \Apre \), que permite
tratar directamente los vértices bajo equidad.

% la idea va a ser intentar dar un poco una intuicion pero mandar a leer pieterman pnuelli y banerjee para entender por que funciona esto
% intuicion podria ser minimizador y maximidzador C y esa idea capaz de acumulacion del obj de rabin ?) no se
Dado que todos los operadores involucrados son monótonos, esta fórmula se puede
evaluar iterativamente sobre una base simbólica, y converge en un número finito
de pasos. Más detalles sobre la correctitud de este algoritmo se pueden
encontrar en \cite{Banerjee} y \cite{Piterman}.

%El resultado es un conjunto \( Z^* \subseteq V \) que representa los vértices desde los cuales el jugador \( \square \) posee una estrategia determinista y sin memoria que garantiza, bajo la hipótesis de juego justo, satisfacer la condición de Rabin especificada.

\subsection*{Complejidad del cálculo de estados ganadores en un PSG con objetivo de Rabin}

Haber podido probar la igualdad en \ref{teocuali} nos permite calcular el
conjunto de estados casi seguramente ganadores para $\cuad$ en $\Gk$ mediante
el algoritmo que se propone en \cite{Banerjee} y presentamos en la subsección
anterior.

Este algoritmo tiene una complejidad de $O(n^2 d!)$ donde $n$ es la cantidad de
vértices en $G$ y $d$ la cantidad de pares en $R$.

Esto quiere decir que podemos calcular el conjunto de estados casi seguramente
ganadores para $\cuad$ en un PSG $\K$ con un objetivo de Rabin $R$ con una
complejidad de $O((n l)^2 d! )$ donde $n$ es la cantidad de estados en $\K$,
$d$ es la cantidad de pares en $R$ y $l = \max \{\abs{V_s} \mid s \in \St \}$
es la cantidad máxima de conjuntos soporte para las acciones que puede haber
desde un estado $s$ en $\K$.

\subsection*{Sintésis de estrategias ganadoras en un PSG con objetivo de Rabin}

El cálculo de desde qué estados gana el jugador $\cuad$ responde parcialmente a
la pregunta de ``¿quién gana?'', pero resulta no de menor importancia el
plantearse ``¿cómo se gana?''. Esta pregunta abarca tanto con qué clase de
estrategias se puede ganar como también podría enfocarse en describir paso a
paso cómo se deberían ver esas estrategias y se asocia a lo que es conocido
como el problema de \textit{sintésis de estrategias}, remarcado como de interés
en la literatura. Con lo demostrado en \ref{teocuali}, también respondemos
parcialmente a esta pregunta por lo siguiente.

El algoritmo simbólico planteado en la subsección anterior y propuesto en
\cite{Banerjee} permite extraer una estrategia sin memoria ganadora para el
jugador $\cuad$ en el juego justo, por lo que tendríamos así una estrategia sin
memoria ganadora en la desrandomización de un juego estocástico politópico con
objetivo de Rabin, y como nosotros mostramos cómo, a partir de una estrategia
en la desrandomización, obtener una estrategia en el juego original, con esto
tenemos la manera de sintetizar una estrategia ganadora en un PSG con objetivo
de Rabin.

% El desarrollo de esta prueba en cierto punto nos permitió abordar la pregunta
% que planteamos como de nuestro interés en el capítulo~\ref{cap:objetivos} sobre
% ``¿quién gana?'' desde un estado. Ahora sería interesante poder abordar la
% pregunta de ``¿cómo se gana?''. Esta preguta se suele asociar a lo que es
% conocido como el problema de sintésis de estrategias. -> esto no es quite like
% that pq no hay aca una verdadera sintesis de estrategias sino que hay algoritmo
% para el calculo de esos estados en el juego desrandomizado. se puede hablar un
% poco de sintesis de estrategia al ver que se crean estrategias md a partir de
% estrategias md

% El algortimo de banerjee dice que se puede extraer una estrategia MD desde él, hay que explicarlo para poder tener esta sección

\hl{¿Cómo represento gráficamente la idea de politopos? ¿Capaz dándoles forma a las acciones?}

\section{Transformar Rabin en alcanzabilidad: la respuesta a varias preguntas}

\subsection{Draft}

La idea es intentar hacer algo al estilo

\begin{theorem}[Reducción de Rabin]
	Sea $\Gk$ un juego estocástico politópico y sea $\Hk$ su interpretación extrema tal como se presenta en \cite{Polytopal}. Si tenemos una propiedad de Rabin $R$, y su respectivo conjunto de estados \textit{almost sure winning} $W_R$ entonces valen las siguientes ecuaciones:

	\begin{align*}
		&\inf_{\piGdiam} \sup_{\piGcuad} \ProbG(R) = \\
		= &\inf_{\piGdiam} \sup_{\piGcuad} \ProbG(\alc W_R) = \\
		= &\inf_{\piHdiamMD} \sup_{\piHcuadMD} \ProbH(\alc W_R) = \\
		= &\sup_{\piHcuadMD} \inf_{\piHdiamMD} \ProbH(\alc W_R) = \\
		= &\sup_{\piGcuad} \inf_{\piGdiam} \ProbG(\alc W_R) = \\
		= &\sup_{\piGcuad} \inf_{\piGdiam} \ProbG(R)
	\end{align*}

\end{theorem}

\begin{proof}
	\begin{align*}
		\inf_{\piGdiam} &\sup_{\piGcuad} \ProbG(R) \\
		&\leq \inf_{\piGdiam} \sup_{\piGcuad} \ProbG(\alc W_R)
		& \text{(por lema \ref{*1})}
		\\
		&\leq \inf_{\piHdiamMD} \sup_{\piHcuadMD} \ProbH(\alc W_R)
		& \text{(por Teorema 1)}
		\\
		&\leq \sup_{\piHcuadMD} \inf_{\piHdiamMD} \ProbH(\alc W_R)
		& \text{(por Teorema 1)}
		\\
		&\leq \sup_{\piGcuad} \inf_{\piGdiam} \ProbG(\alc W_R)
		& \text{(por Teorema 1)}
		\\
		&\leq \sup_{\piGcuad} \inf_{\piGdiam} \ProbG(R)
		& \text{(por lema \ref{**2})}
		\\
		&\leq \inf_{\piGdiam} \sup_{\piGcuad} \ProbG(R)
		& \text{(por prop de sup e inf)}
	\end{align*}
\end{proof}

\begin{lemma}
	\label{*1}
	Sea $\Gk$ la interpretación (?) de un juego estocástico politópico y sea $R$ una propiedad límite y $W_R$ su correspondiente conjunto casi seguramente ganador. Entonces vale que
	$$
		\inf_{\pidiamset} \sup_{\picuadset} \ProbG (R) \leq \inf_{\pidiamset} \sup_{\picuadset} \ProbG (\alc W_R)
	$$
\end{lemma}

\begin{proof}
	Nombremos $\pidiam^*$ a una estrategia tal que $$ \sup_{\picuadset} \Prob_{\Gk, s}^{\picuad, \pidiam^*} (\alc W_P)= \inf_{\pidiamset} \sup_{\picuadset} \ProbG(\alc W_P)$$ (explicar algo más hablado? decir algo sobre su existencia?). Ahora podemos hacer las siguientes deducciones:

	\begin{align*}
		\inf_{\pidiamset} &\sup_{\picuadset} \ProbG(P) \\
		&\leq \sup_{\picuadset} \Prob_{\Gk, s}^{\picuad, \pidiam^*} (P) &\text{(por def. de inf)} \\
		&= \sup_{\picuadset} \Prob_{\Gk, s}^{\picuad, \pidiam^*} (\alc W_P) &\text{(por lema \ref{PMDP-supWR})} \\
		&= \inf_{\pidiamset} \sup_{\picuadset} \ProbG(\alc W_P) &\text{(por def. de $\pidiam^*$)}
	\end{align*}

	y con esto hemos probado el enunciado.
\end{proof}

\begin{lemma}
	\label{PMDP-supWR}
	Sea $\M$ un PMDP, $\pi$ una estrategia en $\M$, $s$ un estado en $\M$, $R$ una propiedad de Rabin y $W_R$ el conjunto casi seguramente ganador de $R$ (en el juego estocástico debería ser, no?), entonces vale que:

	% CAMBIAR WR por conjunto desde el que existe una estr que da prob 1 de ganar

	\hl{¿Cambiar wr por conjunto desde el que eciste una estrategia que da probabilidad 1 de ganar?}

	$$
		\sup_{\pi \in \Pi} \ProbPMDP (R) = \sup_{\pi \in \Pi} \ProbPMDP (\alc W_R)
	$$
\end{lemma}

\begin{proof}
	Por definición de conjunto casi seguramente ganador
	$$
		\Prob_{\M,s}^\pi (R) = \Prob_{\M,s}(\{ \omega \in \paths(s) \mid \inft(\omega) \models R\})
	$$

	Claramente los caminos que hacen que valga la propiedad también son caminos
	desde donde llego a algún estado desde donde existen estrategias para ganar con
	probabilidad 1. Por lo tanto, $\ProbPMDP (R) \leq \ProbPMDP (\alc W_R)$.

	Para ver que vale la igualdad, veremos que existe una estrategia de memoria
	finita $\pi$ que hace que $\ProbPMDP (R) = \sup_{\pi \in \Pi} \ProbPMDP (\alc
		W_R) $. Para ello, consideremos la estrategia sin memoria $\pi_0$ que maximiza
	las probabilidades de alcanzar $W_R$ desde todos los estados $s \in \M$
	(sabemos que esta estrategia existe por \cite{Polytopal, CONDON1992}). A su
	vez, sabemos que para cada estado $s \in W_R$ existe una estrategia $\pi_s$ que
	asegura ganar con probabilidad 1 frente al objetivo $R$.

	Sea entonces $\pi$ la estrategia que primero se comporta como $\pi_0$, hasta
	llegar a un estado $t$ en $W_R$, y a partir de allí $\pi$ se comporta como
	$\pi_t$. Con eso tenemos que:

	\begin{align*}
		\ProbPMDP (R) &= \sum_{t \in W_R} \Prob_{\M,s}^{\pi_0} ((\neg W_R \until t)) \cdot \underbrace{\Prob_{\M,t}^{\pi_t}(R)}_{=1} \\
		&= \sup_{\pi \in \Pi} \ProbPMDP (W_R)
	\end{align*}

	Como $\sup_{\pi \in \Pi} \ProbPMDP (W_R)$ es una cota superior las
	probabilidades para $R$ bajo todas las estrategias, con esto podemos concluir
	la igualdad $\sup_{\pi \in \Pi} \ProbPMDP (R) = \sup_{\pi \in \Pi} \ProbPMDP
		(\alc W_R)$.
\end{proof}

\hl{Esto es lo importante que habría que probar y no está probado.}

\begin{lemma}
	\label{**2}
	Sea $\Gk$ la interpretación (?) de un juego estocástico politópico y sea $R$ una propiedad límite y $W_R$ su correspondiente conjunto casi seguramente ganador. Entonces vale que
	$$
		\sup_{\picuadset} \inf_{\pidiamset} \ProbG (\alc W_R) \leq \sup_{\picuadset} \inf_{\pidiamset}  \ProbG (R)
	$$
\end{lemma}

\hl{¿Agregar una sección hablando de algo de complejidad en PSG con objetivos de Rabin?}
