\chapter{Verificación de modelos con juegos}
~\label{cap:approach}

\section{Sobre la verificación de modelos}

blabla

problema de sintesis de church

idea de la necesidad de juegos y nuestro entendimiento de ellxs

¿Que preguntas nos hacemos cuando trabajamos con juegos? Mas que nada con juegos estocasticos
tenemos las preguntas que se hace Kucera, las que presenta Chatterjee y se puede investigar mas
Capaz esto iría mas en la introducción para platear la idea de trabajo

\section{Cadenas de Markov}

Baier and Katoen

\section{Procesos de Decisión de Markov}

Baier and Katoen + Tesis de l de alfaro

\section{Juegos estocásticos}

\subsection{Una breve historia sobre los juegos estocásticos}

Enfoque mc -> mdp -> sg -> psg

Enfoque juegos deterministas -> juegos estocasticos (Shapley etc)

\subsection{Draft}
Cosas de Kucera y de la parte preliminar del paper de Pedro.

Me gustaría incluir cosas de Kucera/Banerjee y sus ideas de <<1>> val etc.
(concepto de distintas maneras de ganar)

Capaz algo de omega regular stochastic games y qsy

\subsubsection{Objetivos}

distintos tipos y organizacion / ver de que manera presentarlos

\section{Juegos deterministas}

Agg transiciones de texto escrito entre definiciones

\begin{definition}[juego de grafo de 2 jugadores]
	Definimos a un juego de grafo de dos jugadores (2G) como una tupla $G = (V, V_\cuad,V_\diam, E)$ donde $V = V_\cuad \biguplus V_\diam$ es un conjunto de vértices (o estados) particionado en $V_\cuad$ y $V_\diam$, y $E \subseteq (V\times V)$ es una relación que denota el conjunto de aristas (dirigidas) que representan transiciones de un estado a otro del juego.

	Los 2 jugadores son llamados $\cuad$ y $\diam$ y controlan los vértices
	$V_\cuad$ y $V_\diam$, respectivamente.
\end{definition}

\begin{definition}[estrategia sobre un 2G]
	Una estrategia para un jugador $i$ con $i \in \{\cuad, \diam\}$ es una función $\sigma_i: V^*V_i \rightarrow V$ con la restricción de que $\sigma_i(wv) \in E(v)$ para todo $wv \in V^*V_i$.
\end{definition}

\begin{definition}[jugada sobre un 2G]
	Una jugada en un juego de grafo de dos jugadores es una secuencia infinita de vértices $\rho = v_0v_1v_2 \dots \in V^\omega$, donde para todo $i \in \N_0$ tenemos que $v^i \in V$ y $(v^i, v^{i+1}) \in E$.

	Sean $\sigma_\cuad$ y $\sigma_\diam$ un par de estrategias y sea $v_0$ un
	vértice inicial, la jugada que cumple con $\sigma_\cuad$ y $\sigma_\diam$ es la
	única jugada $\rho = v_0 v_1 v_2 \dots$ para la cual cada $i \in \N_0$, si $v_i
		\in V_\cuad$ entonces $v_{i+1} = \sigma_\cuad(v_0 \dots v_i)$, y si $v_i \in
		V_\diam$ entonces $v_{i+1} = \sigma_\diam(v_0 \dots v_i)$.
\end{definition}

\begin{definition}[condición ganadora]
	Una condición ganadora $\varphi$ en un juego de grafo de dos jugadores es un conjunto de jugadas sobre el juego, i.e., $\varphi \subseteq V^\omega$. Usaremos notación LTL para describir conjuntos de jugadas específicos.
\end{definition}

\begin{definition}[regiones ganadoras]
	El jugador $\cuad$ gana el juego de grafo de dos jugadores $G$ para una condición ganadora $\varphi$ desde un vértice $v_0$ so existe una estrategia $\picuad$ tal que para cada $\pidiam$, la jugada $\rho$ que sigue $\picuad$ y $\pidiam$ satisface $\varphi$, i.e., $\rho \in \varphi$.
	La región ganadora $\W \subseteq V$ para el jugador $\cuad$ es el conjunto de vértices desde donde el jugador $\cuad$ gana el juego.
\end{definition}

Agg ejemplo

\section{Comparación de los distintos juegos}

Incluir cuadros comparativos entre juegos // ejemplos para todas las nociones
nuevas explicadas

(Capaz agregar un apéndice con cuadritos de esto ?)

(En general, estaría bueno ver problemas y agregar un apéndice con ejemplos)