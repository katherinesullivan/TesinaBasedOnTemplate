\chapter{Titulo del capitulo}
~\label{cap:example}

En este capitulo hablaremos de la sección~\ref{cap:intro:sec:contributions}
para demostrar teorema~\ref{teorema1}.~\comentario{Acá dejo un comentario}

\section{Titulo de la sección}
~\label{cap:example:sec:title}

Ejemplo de Sección~\seba{Ejemplo de comentario de otra persona}

\section{Teoremas}
~\label{cap:example:sec:theorem}

A continuación, plantearemos el teorema 1~\ref{teorema1}.

\subsection{Teorema 1}
~\label{cap:example:sec:theorem:subsec:1st-theorem}

\begin{lemma}~\label{lema1}
	Enunciado del lema.
\end{lemma}

\begin{proof}
	Prueba del lema.
\end{proof}

\begin{theorem}~\label{teorema1}
	Enunciado del teorema.
\end{theorem}

\begin{proof}
	Prueba del teorema.
\end{proof}

\section{Figuras y tablas}
~\label{cap:example:sec:fig-and-tables}

Ejemplo de Tabla~\ref{table:example}

\begin{center}~\label{table:example}
	\begin{tabular}{||c c c c||}
		\hline
		Col1 & Col2 & Col2 & Col3 \\ [0.5ex]
		\hline\hline
		1 & 6 & 87837 & 787 \\
		\hline
		2 & 7 & 78 & 5415 \\
		\hline
		3 & 545 & 778 & 7507 \\
		\hline
		4 & 545 & 18744 & 7560 \\
		\hline
		5 & 88 & 788 & 6344 \\ [1ex]
		\hline
	\end{tabular}
\end{center}

Ejemplo imagen~\ref{figure:functional}:

\begin{figure}[!ht]
	\centering
	\includegraphics[width=0.4\textwidth]{Imgs/FunctionalWay.jpg}
	\caption{\label{figure:functional} La posta}
\end{figure}

\newpage{}

\section{Algoritmos}
~\label{cap:example:sec:algorithms}

\begin{algorithm}
	\caption{Programin}\label{alg:1}
	\begin{algorithmic}[1]
		\Procedure{programa}{$a, n$}
		\State{$a \gets 0$}
		\For{$i = 0\ to\ n$}
		\State{$a\ |=\ n\ \oplus \ a$}
		\EndFor{}
		\State{\Return{$result$}}
		\EndProcedure{}
	\end{algorithmic}
\end{algorithm}

\begin{algorithm}
	\caption{Unbalanced Program}\label{alg:2}
	\begin{algorithmic}[1]
		\Procedure{Unbalanced\_program}{$public_1, public_2, public_3, secret_1$}
		\State{$public_1 \mathrel{+}= public_2$}
		\If{$public_1 > 20$}\Comment{If irrelevante, pues es público}
		\State{$public_1 \mathrel{+}= public_2 + public_3$}
		\Else{}
		\State{$public_1 \mathrel{+}= public_2$}
		\EndIf{}
		\If{$public_2 > secret_1$}\Comment{If con dependencia de secretos}
		\State{$public_1 \mathrel{+}= public_2$}
		\State{$public_1 \mathrel{+}= public_3$}
		\Else{}
		\State{$public_1 \mathrel{+}= public_2$}
		\State{$public_1 \mathrel{*}= public_3$}
		\EndIf{}
		\State{\Return{$public_1$}}
		\EndProcedure{}
	\end{algorithmic}
\end{algorithm}

Comparamos Algoritmo~\ref{alg:1} y Algoritmo~\ref{alg:2}.