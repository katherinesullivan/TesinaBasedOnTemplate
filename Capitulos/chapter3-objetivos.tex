\chapter{Objetivos $\omega$-regulares}
~\label{cap:objetivos}

% Resulta practico estudiar a los juegos en relacion a algun tipo de objetivo en
% particular (¿por que? agg algo aca - capaz citas van) Como mencionamos antes
% nosotros nos enfocamos en el estudio de juegos estoc polito con objetivos de
% rabin, pero la pregunta es por que? ... kinda that mejor redactado
%puede borrarse sino directamente este parrafo

En este capítulo presentaremos la formalización de los objetivos en juegos,
hablaremos un poco sobre las preguntas que nos planteamos cuando estudiamos
juegos, veremos una clasificación típica de los objetivos $\omega$-regulares y
desarrollaremos sobre la importancia de los objetivos de Rabin y nuestro
interés en su estudio.

\section{Objetivos en juegos y preguntas de investigación}

En la literatura, un objetivo $\phi$ para un juego es definido como un conjunto
de caminos. Para los juegos deterministas nosotros mantendremos esta
definición, pero como para los juegos estocásticos definimos a los caminos como
secuencias alternantes de estados y acciones (lo cual no es muy usual en la
literatura), debemos remarcar que un objetivo en ellos será un conjunto de
secuencias infinitas de estados que se corresponderían a un camino, es decir,
siendo $\G$ un juego estocástico y $\Paths(\G)$ el conjunto de caminos en él,
un objetivo será algún $\phi \subseteq \{\omega' | \exists \omega = (s_1,
	\alpha_1, s_2, \alpha_2, \dots) \in \Paths(\G) \text{ tal que } \omega' = (s_1,
	s_2, \dots)\} = \{(s_1, s_2, \dots) | \forall i \geq 0, \exists \alpha \in
	\Act(s_i) \text{ tal que } \alpha(s_{i+1}) > 0\}$. A lo largo de esta sección
nos seguiremos refiriendo a objetivos como conjuntos de caminos para evitar
tener que ir diferenciando, pero pensamos que es importante recalcar que los
objetivos se plantean sobre conjuntos de secuencias de estados y no son
planteados como dependientes de las acciones tomadas.

Con eso aclarado, un objetivo para el jugador $\cuad$ especifica un conjunto de
caminos que son ganadores para el jugador $\cuad$, y un objetivo para el
jugador $\diam$ especifica un conjunto de caminos ganadores para el jugador
$\diam$. En el caso de juegos de suma cero (que son con los que trabajaremos),
los objetivos de los dos jugadores son estrictamente competitivos, es decir,
uno es el complemento del otro, por eso nos basta con asociarle un objetivo al
juego, que será el asociado al jugador $\cuad$ y se puede deducir que el
objetivo para el jugador $\diam$ será su complemento.

% El hablar de objetivos nos permitirá también hablar de regiones ganadoras. Una
% región ganadora para un objetivo $\phi$ será el conjunto de estados (o
% vértices) desde donde se puede asegurar que el jugador $\cuad$ ganará, es
% decir, estados desde donde existe una estrategia del jugador $\cuad$ tal que
% frente a cada estrategia del jugador $\diam$ genera un camino dentro del
% conjunto $\phi$.

Definir la noción de objetivos nos permitirá hacernos la pregunta que uno
primero asocia con juegos: ``¿quién gana''. Para que esta pregunta pueda ser
respondida en un contexto académico la deberíamos formalizar, y cuando lo
hacemos, en realidad son muchas las preguntas que se pueden pensar sobre
``¿quién gana?''.

Podemos, por ejemplo, preguntarnos ¿desde qué estados podemos asgurarnos que
gana un jugador? Es decir, fijado un estado $s$ nos podemos preguntar si existe
una estrategia para algún jugador tal que frente a cualquier estrategia del
otro jugador puede asegurar que se produce un camino dentro del conjunto
$\phi$. Esta pregunta en la literatura de juegos estocásticos es conocida como
la pregunta cualitativa (y es esta pregunta sobre la cual enfocaremos nuestro
estudio). Se la llama así en contraposición a la otra pregunta interesante que
también nos podemos hacer en los juegos estocásticos, la pregunta cuantitativa:
para un estado $s$, ¿cuál es la probabilidad de que desde $s$ gane algún
jugador?. Esta última pregunta puede ser respondida para todos los estados, aún
cuando no existe una estrategia que asegure un comportamiento ganador frente a
todas las estrtategias del otro jugador. La respuesta a esta pregunta es
conocida como el valor del estado y comentaremos brevemente sobre ella en el
Capítulo~\ref{cap:conclusions}.

El conjunto de estados desde donde el jugador $\cuad$ gana para algún objetivo
$\phi$ será llamado la región ganadora del juego. El aporte más importante de
este trabajo presentará cómo calcular la región ganadora de un juego
estocástico politópico con un tipo particular de objetivo $\omega$-regular: un
objetivo de Rabin.

% \begin{definition}[regiones ganadoras]
% 	El jugador $\cuad$ gana el juego de grafo de dos jugadores $G$ para una condición ganadora $\varphi$ desde un vértice $v_0$ si existe una estrategia $\picuad$ tal que para cada $\pidiam$, la jugada $\rho$ que sigue $\picuad$ y $\pidiam$ satisface $\varphi$, i.e., $\rho \in \varphi$.
% 	La región ganadora $\W \subseteq V$ para el jugador $\cuad$ es el conjunto de vértices desde donde el jugador $\cuad$ gana el juego.
% \end{definition} 
% O ver definicion de respetar estrategias en desrandomización

Una importante subclase de objetivos son los objetivos $\omega$-regulares. Los
objetivos $\omega$-regulares nacen de la extensión de la teoría de lenguajes
regulares al dominio de las palabras infinitas. Mientras que los lenguajes
regulares clásicos describen conjuntos de cadenas finitas reconocibles por
autómatas finitos, los lenguajes $\omega$-regulares operan sobre secuencias
infinitas, capturando comportamientos infinitos. Esta generalización fue
formalizada en los años sesenta con los autómatas de Büchi, que introdujeron
condiciones de aceptación sobre corridas infinitas de autómatas de estado
finito.

Esta subclase de objetivos resulta de importancia en el contexto de la
verificación y síntesis de sistemas reactivos, donde es necesario especificar y
razonar sobre comportamientos que pueden durar indefinidamente. Estos objetivos
permiten expresar propiedades como ``algo bueno ocurre infinitamente a menudo"
o ``algo malo ocurre sólo una cantidad finita de veces".

\section{Clasificación de objetivos $\omega$-regulares}

Existen varios tipos de objetivos $\omega$-regulares, cada uno definido por
diferentes condiciones de aceptación en los autómatas sobre palabras infinitas.
En particular, las siguientes especificaciones de condiciones de aceptación
definen objetivos $\omega$-regulares y son las más estudiadas:

\begin{itemize}
	\item \textbf{Objetivos de alcanzabilidad y seguridad.} Una especificación de alcanzabilidad para un juego $G$ es un conjunto $T \subseteq S$ de estados. La especificación de alcanzabilidad requiere que algún estado en $T$ sea visitado. Así, la especificación de alcanzabilidad $T$ define el conjunto $\Reach(T) = \{(s_0, s_1, s_2, \dots) \in \Paths(G) | \exists k \geq 0, s_k \in T\}$ de caminos ganadores; este conjunto es llamado un objetivo de alcanzabilidad.

	      Una especificación de seguridad para $G$ es también un conjunto $U \subseteq S$
	      de estados, llamados estados seguros. La especificación de seguridad $U$
	      requiere que solo estados en $U$ sean visitados. Formalmente, el objetivo de
	      seguridad definido por $U$ es el conjunto $\Safe(U) = \{(s_0, s_1, \dots) \in
		      \Paths(G) | \forall k \geq 0, s_k \in U\}$ de caminos ganadores. Nótese que
	      alcanzabilidad y seguridad son objetivos duales: $\Safe(U) = \Paths(G)
		      \setminus (\Reach(S \setminus U))$.

	\item \textbf{Objetivos de Büchi y co-Büchi}. Una especificación de Büchi para $G$ es un conjunto $B \subseteq S$ de estados, que son llamados estados de Büchi. La especificación de Büchi requiere que algún estado en $B$ sea visitado infinitamente a menudo. Para un camino $\omega = (s_0, s_1, s_2, \dots)$, escribimos $\Inf(\omega) = \{s \in S | s_k = s \text{ para infinitos } k \geq 0\}$ para el conjunto de estados que ocurren infinitamente a menudo en $\omega$. Entonces, el objetivo de Büchi definido por $B$ es el conjunto $\Buchi(B) = \{\omega \in \Paths(G) | \Inf(\omega) \cap B \neq \emptyset\}$ de caminos ganadores.

	      El dual de una especificación de Büchi es una especificación de co-Büchi $C
		      \subseteq S$, que especifica un conjunto de estados llamados co-Büchi. Esta
	      especificación requiere que los estados fuera de $C$ sean visitados solo una
	      cantidad finita de veces. Formalmente, el objetivo de co-Büchi definido por $C$
	      es el conjunto $\coBuchi(C) = \{\omega \in \Paths(G) | \Inf (\omega) \subseteq
		      C\}$ de caminos ganadores.

	      Cabe destacar también que los objetivos de alcanzabilidad y seguridad pueden
	      ser transformados a objetivos de Büchi y co-Büchi, respectivamente, modificando
	      un poco el juego $G$. Por ejemplo, si el juego $G'$ resulta de $G$ al
	      transformar cada estado $s \in T$ en un estado absorvente, entonces un juego
	      jugado en $G$ con el objetivo de alcanzabilidad $\Reach(T)$ es equivalente a un
	      juego jugado en $G'$ con el objetivo de Büchi, $\Buchi(T)$.

	\item \textbf{Objetivos de Rabin y Streett}. Ahora pasamos a combinaciones booleanas de objetivos de Büchi y co-Büchi. Una especificación de Rabin para el juego $G$ es un conjunto finito $R = \{(E_1,F_1) \dots (E_d, F_d)\}$ de pares de conjuntos de estados, esto es, $E_j \subseteq S$ y $F_j \subseteq S$ para todo $ 1 \leq j \leq d$. Los pares en $R$ son llamados pares de Rabin. Asumimos, sin pérdida de generalidad, que $\bigcup_{1 \leq j \leq d} (E_j \cup F_j) = S$. La especificación de Rabin $R$ requiere que para algun par de Rabin $1 \leq j \leq d$, todos los estados en el conjunto de la izquierda $E_j$ sean visitados solo una cantidad finita de veces, y que algún estado en el conjunto de la derecha $F_j$ sea visitado infinitamente a menudo. Con eso, el objetivo de Rabin definido por $R$ es el conjunto $\Rabin(R) = \{\omega \in \Paths(G) | \exists j \in [1, d], \Inf(\omega) \cap E_j = \emptyset \wedge \Inf(\omega) \cap F_j \neq \emptyset \}$ de conjuntos ganadores. Nótese que el objetivo co-Büchi $\coBuchi(C)$ es igual al objetivo de Rabin con un único par $\Rabin(\{(C,S)\})$ y que el objetivo de Büchi $\Buchi(B)$ es igual al objetivo de Rabin con dos pares $\Rabin(\{(\emptyset, B), (S,S)\})$ (por la asumpción que hicimos sobre la unión de los conjuntos de los pares).

	      Los complementos de los objetivos de Rabin son los objetivos de Streett. Una
	      especificación de Streett para $G$ es también un conjunto $W = \{(E_1, F_1),
		      \dots, (E_d, F_d)\}$ de pares de conjunto de estados $E_j \subseteq S$ y $F_j
		      \subseteq S$ tal que $\bigcup_{1 \leq j \leq d} (E_j \cup F_j) = S$. Los pares
	      en $W$ se llaman pares de Streett. La especificación de Streett $W$ requiere
	      que para cada par de Streett $1 \leq j \leq d$, si algún estado en el conjunto
	      de la derecha $F_j$ es visitado infinitamente a menudo, entonces algún estado
	      en el conjunto de la izquierda $E_j$ es visitado infinitamente a menudo.
	      Formalmente, el objetivo de Streett definido por $W$ es el conjunto
	      $\Streett(W) = \{\omega \in \Paths(G) | \forall j \in [1,d], \Inf(\omega) \cap
		      E_j \neq \emptyset \vee \Inf(\omega) \cap F_j = \emptyset\}$ de caminos
	      ganadores. Nótese que $\Streett(W) = \Paths(G) \setminus \Rabin(W)$.

	\item \textbf{Objetivos de paridad}.Una especificación de paridad para \(G\) consiste en un entero no negativo \(d\) y una función \(p\colon S\to\{0,1,2,\dots,2d\}\), que asigna a cada estado de \(G\) un entero entre \(0\) y \(2d\). Para un estado \(s\in S\), el valor \(p(s)\) se denomina \emph{prioridad} de \(s\). Sin pérdida de generalidad asumimos que \(p^{-1}(j)\neq\emptyset\) para todo \(0<j\le2d\); esto implica que la especificación queda totalmente determinada por la función \(p\) (no es necesario indicar \(d\) explícitamente). El número \(2d+1\) es el número de prioridades de \(p\). La especificación exige que la prioridad mínima de todos los estados visitados infinitamente a menudo sea par. Formalmente, el objetivo de paridad definido por \(p\) es el conjunto \( \Parity(p)=\{\omega\in \Paths(G) | \min\{p(s)\mid s\in \Inf(\omega)\}\text{ es par}\}\) de caminos ganadores. Nótese que su complemento es también un objetivo de paridad, pues \(\Paths(G)\setminus \Parity(p)= Parity(p+1)\), donde \((p+1)(s)=p(s)+1\) para todo \(s\in S\) (si \(p^{-1}(0)=\emptyset\) se usa \(p-1\) en lugar de \(p+1\)). Esta autodualidad de los objetivos de paridad resulta muy conveniente al resolver juegos. Es también interesante notar que los objetivos de Büchi son objetivos de paridad con dos prioridades (siendo \(p^{-1}(0)=B\) y \(p^{-1}(1)=S\setminus B\)) y los objetivos de co-Büchi son objetivos de paridad con tres prioridades (siendo \(p^{-1}(0)=\emptyset\), \(p^{-1}(1)=S\setminus C\) y \(p^{-1}(2)=C\)).

	      Los objetivos de paridad también se llaman objetivos Rabin-chain, pues son un
	      caso especial de objetivos de Rabin: si los conjuntos de una especificación
	      Rabin $R = \{(E_1,F_1),\dots,(E_d,F_d)\}$ forman una cadena \(E_1\subsetneq
	      F_1\subsetneq E_2\subsetneq F_2\subsetneq\cdots\subsetneq E_d\subsetneq F_d\),
	      entonces $\Rabin(R) = \Parity(p)$ para la función de prioridades $p : S \to
		      \{0,1,\dots,2d\}$ que asigna a cada estado en \(E_j\setminus F_{j-1}\) la
	      prioridad \(2j-1\), y a cada estado en \(F_j\setminus E_j\) la prioridad
	      \(2j\), donde \(F_0=\emptyset\). Recíprocamente, dada una función de
	      prioridades $ p : S \to \{0,1,\dots,2d\}$, podemos construir una cadena
	      \(E_1\subsetneq F_1\subsetneq\cdots\subsetneq E_{d+1}\subsetneq F_{d+1}\) de
	      \(d+1\) pares de Rabin tal que \( \Parity(p) =
	      \Rabin(\{(E_1,F_1),\dots,(E_{d+1},F_{d+1})\})\) de la siguiente manera: sea
	      \(E_1=\emptyset\) y \(F_1=p^{-1}(0)\), y para todo \(1\le j\le d+1\) sea
	      \(E_j=F_{j-1}\cup p^{-1}(2j-3)\) y \(F_j=E_j\cup p^{-1}(2j-2)\). Por tanto, los
	      objetivos de paridad son una subclase de los objetivos de Rabin que está
	      cerrada bajo complementación; de ello se sigue que todo objetivo de paridad es
	      a la vez un objetivo de Rabin y un objetivo de Streett. % Su interés radica en que cualquier objetivo \(\omega\)-regular puede convertirse en uno de paridad tomando el producto síncrono del grafo de juego con un autómata de paridad determinista que acepte dicho objetivo.

	\item \textbf{Objetivos de Müller}. Una especificación de Müller para \(G\) es un conjunto \(M\subseteq2^S\) cuyos elementos se llaman \emph{conjuntos de Müller}. La especificación exige que el conjunto de estados visitados infinitamente a lo largo de una trayectoria pertenezca a \(M\); formalmente, el objetivo de Müller definido por \(M\) es $\Muller(M)=\{\omega\in \Paths(G) \mid \Inf(\omega)\in M\}$. Es fácil notar que un objetivo de Rabin es un caso especial de un objetivo de Müller, pero es también cierto que un objetivo de Müller puede ser transformado en un objetivo de Rabin. Se puede encontrar una prueba de esta afirmación que sigue la técnica de \emph{latest appearence record} en la sección 1.4.2 de \cite{AutomataLogicsInfiniteGames}.

	      Capaz agg tal ejemplo en tal juego es un objetivo de X
\end{itemize}

% Los objetivos de Rabin son especialmente importantes porque cualquier lenguaje
% $\omega$-regular puede ser reconocido por un autómata de Rabin. Además, los
% autómatas de Rabin son fundamentales en la teoría de juegos y síntesis de
% controladores, ya que permiten expresar una amplia gama de especificaciones y
% tienen buenas propiedades de cierre bajo operaciones booleanas.

% ¿de dónde salen?

% ¿cuáles son los tipos de objetivos omega regulares más conocidos? agg 1 ejemplo de cada uno

% ¿por qué los objetivos de Rabin son importantes?

\section{La importancia de los objetivos de Rabin}

Con lo presentado en la sección anterior, poedemos deducir algo muy interesante
de los objetivos de Rabin. Mencionamos primero que los objetivos de
alcanzabilidad y seguridad pueden transformarse en objetivos de Büchi y
co-Büchi. Y luego mencionamos que tanto los objetivos de Büchi como de co-Büchi
pueden trasnformarse en objetivos de Rabin (con lo cual también podemos hacer
lo mismo con los objetivos de alcanzabilidad y seguridad). A su vez, también
mencionamos que los objetivos de paridad y de Müller pueden ser transformados a
objetivos de Rabin. Con esto, podemos empezar a deducir lo que se prueba
formalmente con la prueba de Safra en \cite{Safra}: cada objetivo
$\omega$-regular puede ser transformado a un objetivo de Rabin. (Safra lo que
prueba en realidad es que cada automata no determinista de Büchi -de los cuales
se sabe que pueden reconocer todos los lenguajes $\omega$-regular- se puede
trasnformar en un autómata determinista de Rabin).

Esta es la razón principal por la cual decidimos enfocarnos en objetivos de
Rabin en esta tesina. Estudiar juegos con objetivos de Rabin es tan general
como estudiar juegos con cualquier tipo de objetivo $\omega$-regular; cada
resultado que obtengamos vale para juegos con cualquier objetivo
$\omega$-regular.

Por otro lado, también resulta de particular interés el estudio de juegos con
objetivos de Rabin y su dual, objetivos de Streett, porque su forma coincide
con la de las condiciones de equidad (\textit{fairness} en inlglés). Estas
propiedades, que son un tipo un particular de propiedades de vitalidad
(\textit{liveness} en inglés), son de las más clásicas que se solicitan que
tengan los sistemas reactivos, por lo que su estudio tiene implicancias
prácticas concretas directas.

% Por otro lado, resulta de particular interés el estudio de juegos con objetivos
% de Rabin cuando equidad --- ver cosas de maxi --- ver de donde sque lo que
% equidad --- hay algo en strategy improvement for stochastic rabin and streett
% games --- capaz hay algo en banerjee o chaterjee

Entonces, con lo desarrollado se puede entender bien el porqué de la segunda
parte del título de esta tesina: objetivos de Rabin, pero nos falta todavía
adentrarnos en el porqué de la primera parte: juegos estocásticos politópicos.
Eso es lo que haremos en el próximo capítulo.

% PREGUNTAS DE INVESTIGACIÓN?