\section{Procesos de Decisión de Markov Politópicos - PMDPs}
\label{sec:pmdp}

Llamaremos proceso de decisión de Markov politópico y lo indicaremos con las
siglas PMDP a un PSG donde $\St_\cuad = \emptyset$ o $\St_\diam = \emptyset$.
Al igual que en el caso de PSGs, haremos el estudio de PMDP sobre la
interpretación de los mismos que formalizamos de la siguiente manera:

\begin{definition}
	La interpretación de un \textit{proceso de decisión de Markov politópico} (PMDP por sus siglas en inglés) es una tupla $\M = (\St, Act, \theta')$ donde $S$ es un conjunto finito de estados, $Act$ es un conjunto de pares (politopo, distribución) y $\theta': \St \times Act \times \St \rightarrow [0, 1]$ es la función de transición entre estados.
\end{definition}

Entonces, los PMDPs puden ser interpretados como MDPs en donde el número de
acciones posibles a tomar desde un estado puede ser no numerable.

% También introducimos el concepto de \textit{comportamiento} en un PMDP. Un
% comportamiento será una secuencia alternante de estados y conjuntos de próximos
% estados posibles, que refleja un proceso de selección de dos pasos. Desde un
% estado s, primero, el jugador selecciona un politopo y una distribución cuyo
% conjunto soporte será un $V \in V_s$, y luego el próximo estado $s'$ será
% elegido probabilísticamente en base a la distribución seleccionada. Presentamos
% la definición formal de la siguiente manera:

Los caminos, las estrategias, y la medida de probabilidad en los PMDPs se
definen tomando las mismas definiciones de los juegos estocásticos politópicos,
pero presentaremos formalmente a continuación las definiciones de conjuntos
estado-resultado, sub-PMDPs y componentes finales que en cierta forma extienden
a los conceptos de conjuntos estado-acción, sub-MDPs y componentes finales que
se suelen encontrar en la literatura de procesos de decisión de Markov
\cite{AlfaroThesis,BaierKatoen}.

\begin{definition}[Conjuntos estado-resultado y sub-PMDPs]
	Dado un PMDP $\M = (S, Act, \theta')$ un conjunto estado-resultado es un subconjunto $\chi \subseteq \{(s, V') \mid s \in S \wedge V' \in V_s\}$. Un sub-PMDP es un par $(C, D)$, donde $C \subseteq S$ y $D$ es una función que asocia a cada $s \in C$ un conjunto $D(s) \subseteq V_s$ de subconjuntos de estados próximos posibles. Hay una relación uno-a-uno entre sub-PMDPs y conjuntos de estado-acción:

	\begin{itemize}
		\item dado un conjunto estado-resultado $\chi$, denotamos $\sub(\chi) = (C, D)$ al
		      sub-PMDP definido por:
		      \[
			      C = \{s \mid \exists V' . (s, V') \in \chi\} \quad D(s) = \{V' \mid (s, V') \in \chi\}
		      \]

		\item dado un sub-PMDP $(C, D)$, denotamos por $\er(C,D) = \{(s,V') \mid s \in C
			      \wedge V' \in D(s)\}$ al conjunto estado-resultado correspondiente a $(C, D)$.
	\end{itemize}
\end{definition}

Si definimos para cada $V'$, un vértice único nuevo $v_{V'}$ podemos ver que
cada sub-PMDP $(C, D)$ induce una \textit{relación de aristas}: hay una arista
$(s, v_{V'})$ de $s \in C$ a $v_{V'}$ para cada $V' \in D(s)$ y hay una arista
de $(v_{V'}, t)$ de $v_{V'}$ con $V' \in D(s)$ a $t \in \St$ sii es posible ir
de $s$ a $t$ en un paso con probabilidad positiva utilizando una acción cuyo
conjunto resultado sea $V'$. La definición formal es como sigue:

\begin{definition}[Relación de aristas $\rho$]
	Para un sub-PMDP, definimos la relación $\rho_{(C,D)}$ como
	\[
		\rho_{(C,D)} = \{(s, v_{V'}) \mid \exists V' \in D(s)\} \cup \{(v_{V'}, t) \mid t \in V' \}
	\]
\end{definition}

\begin{definition}[Componente final] \label{defEC}
	Un sub-PMDP es una componente final si:

	\begin{itemize}
		\item $V' \subseteq C$ para todo $V'$ tal que existe un $s \in C$ donde $V' \in D(s)$

		\item el grafo $(C \cup \{v_{V'} \mid \exists s \in C : V' \in D(s)\}, \rho_{(C,D)})$
		      es fuertemente conexo.
	\end{itemize}

	%Decimos que una componente final está contenida en un sub-PMDP $(C', D')$ si $er(C, D) \subseteq er(C', D')$; decimos que una componente final $(C, D)$ es maximal en un sub-PMDP $(C', D')$ si no hay ninguna otra componente final $(C'', D'')$ tal que $er(C,D) \subset er(C'', D'') \subseteq (C,D)$.
	Llamaremos $\EC(\M)$ al conjunto de todas las componentes finales en un PMDP
	$\M$.
\end{definition}

Intuitivamente, una componente final representa un conjunto de pares
estado-resultado que, una vez en ellos, es posible quedase allí para siempre si
la estrategia escoge las acciones de manera apropiada. Esta intuición se hará
precisa con los siguientes teoremas.

Antes de enunciar estos teoremas, introducimos una abreviatura para el conjunto
de estados-resultados que ocurren infinitamente a menudo en un camino dado. %y una notación para el conjunto (infinito) de acciones con el mismo conjunto resultado.

\begin{definition}[$\inft$]
	Dado un camino $\omega = (s_0, \alpha_0, s_1, \alpha_1, \dots)$ indicamos por
	\[
		\inft (\omega) = \{(s, V') \mid s_k = s \wedge supp(\alpha_k) = V' \text{ para infinitos } k \in \mathbb{N}_0\}
	\]
	al conjunto de pares estado-resultado que ocurren infinitas veces en él.
\end{definition}

Ahora sí, podemos pasar a presentar las primeras demostraciones:

\begin{theorem}[Estabilidad de componentes finales]
	\label{teoEstabilidadEC}
	Sea $(C, D)$ una componente final. Entonces, para cada estrategia $\pi$ existe una estrategia $\pi'$, que difiere de $\pi$ solo en $C$, tal que:
	\begin{equation} \label{estabEC}
		\Prob^\pi_s(\diam C) = \Prob^{\pi'}_s(\diam C) = \Prob^{\pi'}_s(\inft(\omega) = \er(C,D))
	\end{equation}
	para todo $s \in S$.
\end{theorem}

\begin{proof}
	Considérese una estrategia $\pi'$ definida como sigue para cada secuencia $s_0 \dots s_n$ con $n \geq 0$:

	\begin{itemize}
		\item Si $s_n \in C$, la estrategia asignará probabilidad positiva a una única acción
		      $(K, \mu) \in A(s_n, V')$ para cada conjunto resultado $V'$ (notaremos a esta
		      acción particular con $(K, \mu)_{V'}$), y la probabilidad de elegir cada una de
		      esas acciones se distribuirá de manera uniforme. Es decir,
		      \[
			      \pi'(s_0 \dots s_n)(K, \mu) =
			      \begin{cases}
				      \frac{1}{\abs{D(s_n)}} \quad \text{si } (K, \mu) = (K, \mu)_{V'} \text{ para algún } V' \in D(s); \\
				      0 \qquad \quad \text{en otro caso}
			      \end{cases}
		      \]

		\item Si $s_n \notin C$, la estrategia $\pi'$ coincide con $\pi$, i.e.
		      \[
			      \pi(s_0 \dots s_n)(K, \mu) = \pi'(s_0\dots s_n)(K, \mu) \quad \forall (K, \mu) \in Act
		      \]

		      %debería ser \pi(s_0 \dots s_n)
	\end{itemize}

	La primera igualdad en \ref{estabEC} es una consecuencia del hecho de que $\pi$
	y $\pi'$ coinciden fuera de $C$.

	Para la segunda igualdad, basta con ver que bajo la estrategia $\pi'$ una vez
	que un camino entra a $C$, nunca sale de $C$ ni se elige una acción que no esté
	en $D$. Es más, una vez en $C$ un camino visitará todos los estados de $C$
	infinitamente a menudo con probabilidad 1.
\end{proof}

Este teorema nos permite presentar un corolario que nos será útil al momento de
probar el teorema principal de esta tesina.

\begin{corollary}
	\label{adaptB30}
	Sea $\M$ un PMDP y sea $s$ un estado ganador en él para una condición de Rabin $R = \{(E_1, F_1), \dots, (E_d, F_d)\}$.%, cuya especificación LTL es $\varphi$ (es decir, para toda estrategia $\pi$, $\ProbPMDP(\varphi) = 1$). 
	Entonces, para toda componente final $(C,D)$ alcanzable desde $s$, vale que existe algún $j \in [1, d]$ tal que $C \cap E_j= \emptyset$ y $C \cap F_j \neq \emptyset$.
\end{corollary}

\begin{proof}
	Supongamos que existe una componente final $(C', D')$ alcanzable desde $s$ tal que esta no cumple que existe algún $j \in [1, d]$ tal que $C' \cap E_j= \emptyset$ y $C' \cap F_j \neq \emptyset$.

	Por teorema~\ref{estabEC}, sabemos que existe una estrategia válida $\pi'$ tal
	que permite llegar a $C'$ desde $s$ y quedarse allí para siempre. Podemos
	plantear un camino que sigue esa estrategia $\pi'$, llamemóslo $\omega'$. Ahora
	bien, entonces también por como presentamos que será esa estrategia $\pi'$ en
	el teorema~\ref{estabEC}, sabemos que $\Inf(\omega') = C$. Pero, a su vez
	sabemos que ese camino debe cumplir con la condición de Rabin, es decir,
	sabemos que existe algún $j \in [1, d]$ tal que $\Inf(\omega') \cap E_j =
		\emptyset$ y $\Inf(\omega') \cap F_j \neq \emptyset$, lo que quiere decir que
	existe algún $j \in [1, d]$ tal que $C' \cap E_j= \emptyset$ y $C' \cap F_j
		\neq \emptyset$ y tendríamos una contradicción.

	Esta contradicción viene de suponer que existe tal comopnente final. Con lo que
	probamos lo que queríamos.
\end{proof}

El próximo resultado establece que, para cualquier estado inicial y cualquier
estrategia (de memoria finita), un camino terminará con probabilidad 1 en una
componente final. Esta es la razón detrás del nombre ``componente final".

\begin{theorem}[teorema fundamental de las componentes finales] \label{teoFundEC}
	Sea $\M$ un PMDP. Para todo $s \in S$, toda estrategia $\pi$ de memoria finita,
	$$\Prob^\pi_{\M, s}(\{ \omega \in \Paths(s) \mid \sub(\inft(\omega)) \text{ es una componente final}\}) = 1$$
\end{theorem}

\begin{proof}
	Consideremos un sub-PMDP $(C,D)$ que no sea una componente final y sea $\Paths_s^{(C,D)} = \{\hat{\omega} \in \Paths_s \mid \inft(\hat{\omega}) = \er(C, D)\}$ el conjunto de caminos cuyo conjunto de pares estado-resultado que se repiten infinitas veces en él forman el sub-PMDP $(C,D)$.
	%(otra forma de decirlo capaz podría ser: conjunto de caminos que se estabilizan en el sub-PMDP $(C,D)$).

	Si podemos mostrar que
	\begin{equation}\label{omegaEnOmega0}
		\Prob^\pi_{\M, s}(\{ \omega \in \Paths(s) \mid \omega \in \Paths_s^{(C,D)}\}) = 0
	\end{equation}
	como $(C,D)$ es un sub-PMDP cualquiera y como hay una cantidad finita de sub-PMDPs en $\M$, esto es lo mismo que mostrar que
	$$\Prob^\pi_{\M, s}(\{ \omega \in \Paths(s) \mid \sub(\inft(\omega)) \text{ es una componente final}\}) = 1$$.
	Veamos que vale \ref{omegaEnOmega0}, dividiendo en casos según cuál es la
	condición de la definición \ref{defEC} que no se cumple para $(C,D)$:

	\begin{itemize}
		\item Primero, asumamos que existe un $(t, V') \in \er(C,D)$ tal que $V' \nsubseteq
			      C$.

		      %Vamos a usar un argumento similar al de la prueba de la primera contención en el intento anterior.

		      %Para cada $(K, \mu) \in A(t, V')$ podemos definir $r_{(K, \mu)} = \sum_{u \in C} \mu(u)$, la probabilidad de quedarnos en $C$ eligiendo la acción $(K, \mu)$, y sabemos que $r_{(K, \mu)} < 1$. 

		      Sabemos que cada camino en $\Paths_s^{(C,D)}$ toma el par estado-resultado $(t,
			      V')$ infinitas veces. Llamemos $I$ al conjunto de índices infinito que
		      representa los momentos en los que se visita $(t, V')$. Indiquemos con $\mu_i$
		      a la distribución elegida en el momento $i \in I$ y definamos como $r_i =
			      \sum_{u \in C} \mu_i(u)$ a la probabilidad de quedarnos en $C$ en el momento $i
			      \in I$ (que en cada caso será menor a 1 porque $V' \nsubseteq C$). %y cada $\mu$ tal que existe un par $(K, \mu)$ en $A(t, V')$ $\mu$ asigna probabilidad positiva a cada $v \in V'$). %y llamemos al evento de q

		      Como $\pi$ es de memoria finita sucederá que $\pi$ solo puede elegir una
		      cantidad finita de acciones (y, por lo tanto, distribuciones) distintas desde
		      $t$. Esto hace que el conjunto $R = \{r_i \mid i \in I\}$ tenga un máximo,
		      llamémoslo $r$.

		      Para que valga que $\omega$ esté en $\Paths_s^{(C,D)}$ tiene que valer que en
		      infinitos momentos $i$ nos quedemos en $C$. Entonces que vale que $\Prob_s^\pi
			      (\omega \in \Paths_s^{(C,D)}) < r^k$ para todo $k > 0$ natural. Como sabemos
		      que $r < 1$, tenemos que $\Prob^\pi_{\M, s}(\{ \omega \in \Paths(s) \mid \omega
			      \in \Paths_s^{(C,D)}\}) = 0$.

		      %Es decir para un subconjunto infinito $J \subseteq I$, vale que: $$\Prob_{s}^{\pi}(\omega \in \Omega_s^{(C,D)}) \leq \prod_{i \in J} r_i$$

		      %Es decir que queremos ver que, si llamamos $C_n$ al evento de quedarnos en $C$ en el momento $n$,

		      %$\Prob(\limsup_{n \to \infty} C_n) = 1$. Por el segundo teorema de Borel-Cantelli sabemos que esto pasa si $\sum_{n>1} \Prob(C_n)$ diverge.
		      %Podemos probar que $\sum_{n>1}\Prob(C_n)$ diverge porque existe un 

		      %Entonces, tenemos que $\Prob_s^{\pi}(\omega \in \Omega_s^{(C,D)}) = \Pi_{i \in I} $

		      %$DistInf \subseteq A(t, V')$ al conjunto de pares $(K, \mu)$ que se toman en los infinitos momentos que se visita $(t, V')$. Entonces, tenemos que         $$\Prob_s^{\pi}(\omega \in \Omega_s^{C,D}) < \Pi_{(K, \mu) \in DistInf} r_{(K, \mu)}$$ Como $\pi$ es de memoria finita, existirá un conjunto infinito $Ig \subseteq DistInf$

		      %Como cada camino en $\Omega_s^{(C,D)}$ toma el par estado-resultado $(t, V')$ infinitas veces, tenemos que $\Prob_s^{\pi}(\omega \in \Omega_s^{C,D}) < \Pi_{(K, \mu) \in A(t, V')} r_{(K, \mu)}$ para todo $k \in \mathbb{N}$ y como $r < 1$, tenemos que 

		      %(No podemos decir que existe un máximo $r_{(K, \mu)}$ porque $A(s, V')$ no es un conjunto finito (ni tampoco que existe una cota superior) así que no sabría si está bien / si lo puedo formalizar más / si puedo definir algún $r$)

		\item Si no, asumamos que existen $t_1, t_2 \in C$ tales que no hay camino de $t_1$ a
		      $t_2$ en $(C \cup \{v_V' \mid \exists s \in C : V' \in D(s)\}, \rho(C,D))$.

		      La falta de camino de $t_1$ a $t_2$ en $(C, \rho_{(C,D)})$ implica que para
		      cada subsecuencia $s_m V_m s_{m+1} ... s_n$ de camino en $\Paths_s^{(C,D)}$ que
		      vaya de $s_m=t_1$ a $s_n=t_2$, hay 2 opciones:

		      \begin{enumerate}
			      \item existe un $j \in [m+1, n-1]$ tal que $s_j \notin C$. Como cada camino en
			            $\Paths_s^{(C,D)}$ contiene infinitas subsecuencias de $t_1$ a $t_2$ y tenemos
			            una cantidad finita de estados, sabemos que habrá una cantidad infinita de
			            $s_j$ iguales. Pero si infinitas veces se toma un estado $s_j$ entonces, $s_j
				            \in C$, lo que contradice la hipótesis anterior. Absurdo.

			      \item existe un $j \in [m, n-1]$ tal que $V_j \notin D(j)$. Como cada camino en
			            $\Paths_s^{(C,D)}$ contiene infinitas subsecuencias de $t_1$ a $t_2$ y tenemos
			            una cantidad finita de conjuntos resultado, sabemos que habrá una cantidad
			            infinita de $V_j$ iguales, con lo que $V_j \in D(j)$, lo que contradice la
			            hipótesis anterior. Absurdo
		      \end{enumerate}
		      Con lo que arribamos a que $\Prob^\pi_{\M, s}(\{ \omega \in \Paths(s) \mid \omega \in \Paths_s^{(C,D)}\}) = 0  $
	\end{itemize}

\end{proof}

Esto nos permite definir el siguiente corolario útil y clásico en los procesos
de decisión de Markov para el análisis de condiciones de Rabin.

\begin{corollary}
	Sea $\M$ un PMDP, $s$ un estado en él y sea $\pi$ una estrategia de memoria finita en él. Una condición de Rabin $R = \{ (E_1, F_1), \dots, (E_d, F_d)\}$ se satisface desde $s$, siguiendo la estrategia $\pi$ con probabilidad 1 si y solo si para cada componente final $U$ alcanzable desde $s$, existe un $j \in \{ 1, \dots, d\}$ tal que $U \cap E_j = \emptyset$ y $U \cap F_j \neq \emptyset$.
\end{corollary}

\textbf{Nota sobre la limitación a estrategias finitas del Teorema~\ref{teoFundEC}}

Ahora bien, una pregunta muy natural que surge es "¿por qué nos restringimos a
estrategias de memoria finita en este último teorema?"

En un principio la idea fue poder probar ambos teoremas en su caso general, sin
tener que limitarnos en la clase de las estrategias utilizadas. La idea para
probar esto esencialmente fue intentar seguir como modelo las pruebas
realizadas para MDPs en \cite{AlfaroThesis,BaierKatoen}. Esto en el caso del
teorema~\ref{teoEstabilidadEC} vino sin complicaciones, pero en el caso del
teorema~\ref{teoFundEC} devino en problemas, puesto que la finitud de acciones
salientes de un estado probó ser una hipótesis fundamental.

La idea de prueba consiste en proponer un sub-PMDP $(C, D)$ arbitrario que no
cumple la definición~\ref{defEC} de componente final, dividir el análisis por
casos de cómo no se cumple la definición y mostrar que, en cada caso, la
probabilidad de que un camino genere un sub-PMDP como el propuesto es 0.

Remitiéndonos a nuestra prueba, vemos que en el primer ítem el argumento está
explicítamente respaldado en el hecho de que $\pi$ es de memoria finita. De
esta manera, se puede decir que el conjunto de las distintas probabilidades de
quedarse en $C$ en los momentos $i$ es finito, por lo que tiene un máximo $r$,
lo que permite acotar la productoria $\Pi_{i \in I} \ r_i$ por $\lim_{k \to
		\infty} r^k$, que sabemos que converge a 0 puesto que $r < 1$.

Si no nos restringimos a estrategias de memoria finita no podemos garantizar
esto. Sabemos que la productoria va a estar acotada inferiormente por 0 y
superiormente por 1, pero si tengo una cantidad infinita de $r_i$'s distintos,
aún cuando cada uno de ellos es menor a $1$, la productoria bien podría no
converger a $0$, como se puede ver en el siguiente caso.

Consideremos un PMDP en donde tenemos dos estados $t$ y $s$. Desde $t$, existen
acciones $\mu$ de la forma $\mu(s) = \frac{1}{k^2}$, $\mu(t) = 1 -
	\frac{1}{k^2}$, por lo que se puede definir una estrategia de memoria infinita
$\pi$ a partir de la cantidad de $t$ que se encuentran en el camino de la
siguiente manera:

\begin{align*}
	&\pi(\omega t)(s) = \frac{1}{(\#_t(\omega))^2} \\
	&\pi(\omega t)(t) = 1- \frac{1}{(\#_t(\omega))^2}
\end{align*}

donde $\#_t$ se define inductivamente sobre caminos de la siguiente manera:

\begin{align*}
	&\#_t(\emptyset) = 0 \\
	&\#_t(\omega V' t) = \#_t(\omega) + 1 \\
	&\#_t(\omega V' s) = \#_t(\omega)
\end{align*}

siendo $\omega$ un camino, $\emptyset$ el camino vacío y $s \neq t$.

Si suponemos que $s$ no forma parte de $C$, tenemos que la probabilidad de
quedarnos en $C$ sería $\Pi_{i \geq 0} 1-(\frac{1}{i^2})$, que sabemos que
converge a $\frac{1}{2}$.

\begin{figure}[h]
	\centering
	\begin{tikzpicture}[>=Stealth, node distance=3.5cm, font=\scriptsize]

		% Styles for different shapes
		\tikzset{
			state/.style={draw, circle, minimum size=1.2cm, font=\normalsize},
			square state/.style={draw, rectangle, minimum size=1.2cm, font=\normalsize},
			diamond state/.style={draw, diamond, aspect=1.5, minimum size=1.5cm, font=\normalsize}
		}

		% Nodes
		\node[state] (t) at (0,0) {$t$};
		\node[state] (s) at (4,0) {$s$};

		% Transitions
		\draw[->] (t) to[bend left=20] node[above] {$1 - \frac{1}{i^2}$} (s);
		\draw[->] (t) edge[loop left] node[left] {$\frac{1}{i^2}$} (t);
	\end{tikzpicture}
	\caption{Interpretación del PDMDP con la estrategia $\pi$ fijada.}
	\label{fig:pmdpcontraej}
\end{figure}

Ahora bien, es cierto que esta premisa se basa en el hecho de que $s$ no esté
en $C$, que no es algo que podamos afirmar a priori, y que esto que mostramos
solo implica que este método de prueba no es útil para probar el
teorema~\ref{teoFundEC} para cualquier tipo de estrategias; no significa que no
sea válido.

Esta limitación a la que nos tuvimos que atener aquí, sin embargo, no
constituyó una limitación a la prueba del teorema principal sobre regiones
ganadoras que mostramos más adelante, porque pudimos reducir lo que
necesitabamos para su prueba al corolario~\ref{adaptB30}.

% ES PMDP CHE
% Consideremos un juego estocástico politópico donde tenemos un vértice $t$ y un
% vértice $s$. Desde $t$, existen acciones $\mu$ de la forma $\mu(s) =
% 	\frac{1}{k^2}$, $\mu(t) = 1 - \frac{1}{k^2}$, por lo que se puede definir una
% estrategia de memoria infinita $\pi$ a partir de la cantidad de $t$ que se
% encuentran en el camino de la siguiente manera:

% \begin{align*}
% 	&\pi(\omega t)(s) = \frac{1}{(\#_t(\omega))^2} \\
% 	&\pi(\omega t)(t) = 1- \frac{1}{(\#_t(\omega))^2}
% \end{align*}

% donde $\#_t$ se define inductivamente sobre caminos de la siguiente manera:

% \begin{figure}[h]
% 	\centering
% 	\begin{tikzpicture}[>=Stealth, node distance=3.5cm, font=\scriptsize]

% 		% Styles for different shapes
% 		\tikzset{
% 			state/.style={draw, circle, minimum size=1.2cm, font=\normalsize},
% 			square state/.style={draw, rectangle, minimum size=1.2cm, font=\normalsize},
% 			diamond state/.style={draw, diamond, aspect=1.5, minimum size=1.5cm, font=\normalsize}
% 		}

% 		% Nodes
% 		\node[square state] (t) at (0,0) {$t$};
% 		\node[diamond state] (s) at (4,0) {$s$};

% 		% Transitions
% 		\draw[->] (t) to[bend left=20] node[above] {$1 - \frac{1}{i^2}$} (s);
% 		\draw[->] (t) edge[loop left] node[left] {$\frac{1}{i^2}$} (t);
% 	\end{tikzpicture}
% 	\caption{Interpretación del juego estocástico con la estrategia $\pi$ fijada.}
% 	\label{fig:mi-figura}
% \end{figure}

% \begin{align*}
% 	&\#_t(\emptyset) = 0 \\
% 	&\#_t(\omega V' t) = \#_t(\omega) + 1 \\
% 	&\#_t(\omega V' s) = \#_t(\omega)
% \end{align*}

% donde $\omega$ es un camino, $\emptyset$ representa el camino vacío y $s \neq t$.

% Si suponemos que $s$ no forma parte de $C$, tenemos que la probabilidad de quedarnos en $C$ sería
% $\Pi_{i \geq 0} 1-(\frac{1}{i^2})$, que sabemos que converge a $\frac{1}{2}$.

% \hl{No me gustan mucho estos siguientes párrafos, pero no sé muy bien cómo decir lo que quiero, y siento que vos habías dicho algo más interesante sobre esto.}

% sólo se garantiza para estrategias de memoria finita, debido a la dependencia
% de la prueba en la existencia de una cota superior para la probabilidad de
% permanecer en $C$ (el valor $r$), que sólo puede asegurarse cuando el número de
% distribuciones posibles es finito. En el caso de estrategias de memoria
% infinita, la secuencia de probabilidades podría no estar acotada uniformemente
% por debajo de 1, y la productoria correspondiente podría no converger a 0, como
% se ilustra en el ejemplo anterior. Por lo tanto, la generalización del
% resultado a estrategias de memoria infinita requeriría técnicas de prueba
% diferentes o condiciones adicionales sobre la estructura del PMDP o las
% estrategias consideradas.

% La prueba del teorema \ref{teoFundEC} está inspirada en las pruebas realizadas
% para MDPs (véase \cite{BaierKatoen} \cite{AlfaroThesis}). En ellas, la prueba
% consiste en proponer un sub-PMDP $(C, D)$ arbitrario que no cumple la
% definición \ref{defEC}, dividir el análisis por casos de cómo no se cumple la
% definición de componente final y mostrar que en cada caso la probabilidad de
% que un camino que siga la estrategia del enunciado genere un sub-PMDP como el
% propuesto es 0.

% Si nos remitimos a la prueba que mostramos en \ref{teoFundEC}, vemos que el
% método de prueba se ajusta bien a nuestro caso hasta llegar al primer ítem
% donde el argumento está explícitamente respaldado en el hecho de que $\pi$ es
% de memoria finita. De esta manera, se puede decir que el conjunto de las
% distintas probabilidades de quedarse en $C$ en los momentos $i$ tiene un
% máximo, $r$, lo que permite acotar la productoria $\Pi_{i \in I} \ r_i$ por
% $\lim_{k \to \infty} r^k$, que sabemos que converge a 0, puesto que $r < 1$.

% Si no nos restringimos a estrategias de memoria finita no podemos garantizar
% esto. Sabemos que la productoria va a estar acotada inferiormente por 0 y
% superiormente por 1, pero bien podría no converger a 0, sino a algún otro
% número. Por ejemplo, veamos lo siguiente:

% Ahora bien, es importante aclarar que esto no constituye de ninguna manera un
% contraejemplo, además de que faltaría formalidad en su presentación, resulta
% que plantear $s \notin C$ siendo que tengo probabilidad positiva de visitarlo
% cada vez no tiene mucho sentido. (Esto no sé que tanto es así).

% En cualquier caso, podría ser una muestra de que la aplicación de este fórmula
% de prueba no es útil \textit{directamente} para probar el teorema para
% estrategias de memoria infinita. Pero resaltamos el directamente porque la
% productoria constituye una cota a la probabilidad de que $\omega \in
% 	\Paths_s^{(C,D)}$, no es directamente el valor de la probabilidad.

